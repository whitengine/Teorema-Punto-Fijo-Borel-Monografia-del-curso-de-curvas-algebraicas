%\documentclass[english,10pt,draft]{amsart}
%\documentclass[english,10pt]{amsart}

\documentclass[spanish,10pt]{amsart}

%\usepackage[english]{babel}
\usepackage[spanish]{babel}

\relpenalty=9999
\binoppenalty=9999
%\mathsurround=1pt %espacio antes y después de una fórmula en el texto


%\usepackage[matrix,arrow]{xy}
%\xyoption{all}

\usepackage{amscd,amssymb,amsfonts,amsmath}
\usepackage{tikz-cd}

\usepackage{bm}
\usepackage{graphics}
\usepackage{epsfig}
\usepackage{mathrsfs}
\usepackage{xcolor}
\DeclareMathAlphabet{\mathscrbf}{OMS}{mdugm}{b}{n}

\usepackage{epigraph}

\definecolor{violet}{rgb}{0.0,0.2,0.7}
\definecolor{rouge2}{rgb}{0.8,0.0,0.2}
\usepackage{hyperref}
\hypersetup{
    %bookmarks=true,         % show bookmarks bar?
    unicode=false,          % non-Latin characters in Acrobat’s bookmarks
    pdftoolbar=true,        % show Acrobat’s toolbar?
    pdfmenubar=true,        % show Acrobat’s menu?
    pdffitwindow=false,     % window fit to page when opened
    pdfstartview={FitH},    % fits the width of the page to the window
    pdftitle={},    % title
    pdfauthor={},     % author
    colorlinks=true,       % false: boxed links; true: colored links
   linkcolor=violet,          % color of internal links
    citecolor=rouge2,        % color of links to bibliography
    filecolor=black,      % color of file links
    urlcolor=cyan}           % color of external links


%\renewcommand{\thepart}{\Roman{part}}
\setcounter{tocdepth}{1} %tableofcontents

\unitlength=1cm

\usepackage[text={6.7in,9.2in},centering]{geometry}

\makeatletter
\renewcommand\subsection{\@startsection{subsection}{2}%
  \z@{.5\linespacing\@plus.7\linespacing}{-.5em}%
  %{\normalfont\scshape}}
  %{\normalfont\itshape}}
  {\normalfont\sffamily}}
\makeatother

%\renewcommand{\baselinestretch}{1.1}
%\renewcommand{\baselinestretch}{1.2}


\newcommand{\cqfd}{\hfill $\square$}
\newcommand{\ie}{\textit{i.e. }}

\newcommand{\Hilb}{\textup{Hilb}}
\newcommand{\Pic}{\textup{Pic}}
\newcommand{\Alb}{\textup{Alb}}
\newcommand{\Spec}{\textup{Spec}}
\newcommand{\Proj}{\textup{Proj}}
\newcommand{\Supp}{\textup{Supp}}
\newcommand{\Exc}{\textup{Exc}}
\newcommand{\Rat}{\textup{RatCurves}}
\newcommand{\RatCurves}{\textup{RatCurves}}
\newcommand{\RC}{\textup{RatCurves}^n}
\newcommand{\Univ}{\textup{Univ}}
\newcommand{\Chow}{\textup{Chow}}
\newcommand{\Sing}{\textup{Sing}}
\newcommand{\CS}{\textup{CS}}
\newcommand{\codim}{\textup{codim}}
\newcommand{\Res}{\textup{Res}}
\newcommand{\modulo}{\textup{mod}}




%\let \cedilla =\c
%\renewcommand{\c}[0]{{\mathbb C}}
%\newcommand{\p}[0]{{\mathbb P}}


\newcommand{\Nef}{\textup{Nef}}
%\renewcommand{\Big}{\textup{Big}}
\renewcommand{\div}{\textup{div}}
\newcommand{\Div}{\textup{Div}}
\newcommand{\Amp}{\textup{Amp}}
\newcommand{\Pef}{\textup{Pef}}
\newcommand{\Eff}{\textup{Eff}}
\newcommand{\NS}{\textup{N}^1}
\newcommand{\N}{\textup{N}}
\newcommand{\Z}{\textup{Z}}
\renewcommand{\H}{\textup{H}}
\newcommand{\h}{\textup{h}}
\newcommand{\discrep}{\textup{discrep}}

\newcommand{\NE}{\overline{\textup{NE}}}
\newcommand{\Hol}{\textup{Hol}}

\newcommand{\into}{\hookrightarrow}
\newcommand{\map}{\dashrightarrow}
\newcommand{\lra}{\longrightarrow}

\renewcommand{\le}{\leqslant}
\renewcommand{\ge}{\geqslant}
\newcommand{\Diff}{\textup{Diff}}
\newcommand{\D}{\Delta}
\renewcommand{\d}{\delta}
\newcommand{\G}{\Gamma}
\newcommand{\Fix}{\textup{Fix}}
\newcommand{\Bs}{\textup{Bs}}
\newcommand{\mult}{\textup{mult}}
\newcommand{\Mob}{\textup{Mob}}
\newcommand{\B}{\textup{B}}

\newcommand{\bA}{\textbf{A}}
\newcommand{\bB}{\textup{\textbf{A}}}
\newcommand{\bC}{\textup{\textbf{C}}}
\newcommand{\bD}{\textbf{D}}
\newcommand{\bE}{\textbf{E}}
\newcommand{\bF}{\textbf{F}}
\newcommand{\bG}{\textbf{G}}
\newcommand{\bH}{\textbf{H}}
\newcommand{\bI}{\textbf{I}}
\newcommand{\bJ}{\textbf{J}}
\newcommand{\bK}{\textbf{K}}
\newcommand{\bL}{\textbf{L}}
\newcommand{\bM}{\textbf{M}}
\newcommand{\bN}{\textbf{N}}
\newcommand{\bO}{\textbf{O}}

%\newcommand{\bP}{\mathbb{P}}
\newcommand{\bQ}{\mathbb{Q}}
%\newcommand{\bZ}{\mathbb{Z}}

\newcommand{\bR}{\textup{\textbf{R}}}
\newcommand{\bS}{\textbf{S}}
\newcommand{\bT}{\textbf{T}}
\newcommand{\bU}{\textbf{U}}
\newcommand{\bV}{\textbf{V}}
\newcommand{\bW}{\textbf{W}}
\newcommand{\bX}{\textbf{X}}
\newcommand{\bY}{\textbf{Y}}
\newcommand{\bZ}{\textbf{Z}}
\renewcommand{\bB}{\textbf{B}}
\newcommand{\bP}{\textbf{P}}


\newcommand{\cA}{\mathcal{A}}
\newcommand{\cB}{\mathcal{B}}
\newcommand{\cC}{\mathcal{C}}
\newcommand{\cD}{\mathcal{D}}
\newcommand{\cE}{\mathcal{E}}
\newcommand{\cF}{\mathcal{F}}
\newcommand{\cG}{\mathcal{G}}
\newcommand{\cH}{\mathcal{H}}
\newcommand{\cI}{\mathcal{I}}
\newcommand{\cJ}{\mathcal{J}}
\newcommand{\cK}{\mathcal{K}}
\newcommand{\cL}{\mathcal{L}}
\newcommand{\cM}{\mathcal{M}}
\newcommand{\cN}{\mathcal{N}}
\newcommand{\cO}{\mathcal{O}}
\newcommand{\cP}{\mathcal{P}}
\newcommand{\cQ}{\mathcal{Q}}
\newcommand{\cR}{\mathcal{R}}
\newcommand{\cS}{\mathcal{S}}
\newcommand{\cT}{\mathcal{T}}
\newcommand{\cU}{\mathcal{U}}
\newcommand{\cV}{\mathcal{V}}
\newcommand{\cW}{\mathcal{W}}
\newcommand{\cX}{\mathcal{X}}
\newcommand{\cY}{\mathcal{Y}}
\newcommand{\cZ}{\mathcal{Z}}

\newcommand{\frA}{\mathfrak{A}}
\newcommand{\frB}{\mathfrak{B}}
\newcommand{\frC}{\mathfrak{C}}
\newcommand{\frD}{\mathfrak{D}}
\newcommand{\frE}{\mathfrak{E}}
\newcommand{\frF}{\mathfrak{F}}
\newcommand{\frG}{\mathfrak{G}}
\newcommand{\frH}{\mathfrak{H}}
\newcommand{\frI}{\mathfrak{I}}
\newcommand{\frJ}{\mathfrak{J}}
\newcommand{\frK}{\mathfrak{K}}
\newcommand{\frL}{\mathfrak{L}}
\newcommand{\frM}{\mathfrak{M}}
\newcommand{\frN}{\mathfrak{N}}
\newcommand{\frO}{\mathfrak{O}}
\newcommand{\frP}{\mathfrak{P}}
\newcommand{\frQ}{\mathfrak{Q}}
\newcommand{\frR}{\mathfrak{R}}
\newcommand{\frS}{\mathfrak{S}}
\newcommand{\frT}{\mathfrak{T}}
\newcommand{\frU}{\mathfrak{U}}
\newcommand{\frV}{\mathfrak{V}}
\newcommand{\frW}{\mathfrak{W}}
\newcommand{\frX}{\mathfrak{X}}
\newcommand{\frY}{\mathfrak{Y}}

%\renewcommand{\frm}{\mathfrak{m}} % accolades dans xymatrix

\newcommand{\frm}{\mathfrak{m}}
\newcommand{\frf}{\mathfrak{f}}
\newcommand{\frg}{\mathfrak{g}}
\newcommand{\frt}{\mathfrak{t}}


\newcommand{\sA}{\mathscr{A}}
\newcommand{\sB}{\mathscr{B}}
\newcommand{\sC}{\mathscr{C}}
\newcommand{\sD}{\mathscr{D}}
\newcommand{\sE}{\mathscr{E}}
\newcommand{\sF}{\mathscr{F}}
\newcommand{\sG}{\mathscr{G}}
\newcommand{\sH}{\mathscr{H}}
\newcommand{\sI}{\mathscr{I}}
\newcommand{\sJ}{\mathscr{J}}
\newcommand{\sK}{\mathscr{K}}
\newcommand{\sL}{\mathscr{L}}
\newcommand{\sM}{\mathscr{M}}
\newcommand{\sN}{\mathscr{N}}
\newcommand{\sO}{\mathscr{O}}
\newcommand{\sP}{\mathscr{P}}
\newcommand{\sQ}{\mathscr{Q}}
\newcommand{\sR}{\mathscr{R}}
\newcommand{\sS}{\mathscr{S}}
\newcommand{\sT}{\mathscr{T}}
\newcommand{\sU}{\mathscr{U}}
\newcommand{\sV}{\mathscr{V}}
\newcommand{\sW}{\mathscr{W}}
\newcommand{\sX}{\mathscr{X}}
\newcommand{\sY}{\mathscr{Y}}
\newcommand{\sZ}{\mathscr{Z}}

\newcommand{\sbfE}{\mathscrbf{E}}
\newcommand{\sbfG}{\mathscrbf{G}}
\newcommand{\sbfH}{\mathscrbf{H}}
\newcommand{\sbfL}{\mathscrbf{L}}
\newcommand{\sbfN}{\mathscrbf{N}}

\newcommand{\art}{\textup{Art}}
\newcommand{\ens}{\textup{Ens}}
\newcommand{\sch}{\textup{Sch}}
\newcommand{\Der}{\textup{Der}}



\newtheorem{thm}{Teorema}[section]
\newtheorem*{theorema}{Teorema A}
\newtheorem*{theoremb}{Teorema B}
\newtheorem*{theoremc}{Teorema C}
\newtheorem*{theoremd}{Teorema D}
\newtheorem*{theoreme}{Teorema E}
\newtheorem{maintheorem}[thm]{Teorema Principal}
\newtheorem{question}[thm]{Pregunta}
\newtheorem{lemma}[thm]{Lema}
\newtheorem{corollary}[thm]{Corolario}
\newtheorem{corollaries}[thm]{Corolarios}
\newtheorem{proposition}[thm]{Proposición}
\newtheorem{criteria}[thm]{Criterio}
\newtheorem{conjecture}[thm]{Conjectura}
\newtheorem{principle}[thm]{Principio}
\newtheorem{complement}[thm]{Complemento}
\newtheorem{warning}[thm]{Cuidado}

\newtheorem*{thm*}{Teorema}
\theoremstyle{definition}
\newtheorem{definition}[thm]{Definición}
\newtheorem{condition}[thm]{Condición}
\newtheorem{say}[thm]{}
\newtheorem{hint}[thm]{Hint}
\newtheorem{trick}[thm]{Truco}
\newtheorem{exercise}[thm]{Ejercicio}
\newtheorem{problem}[thm]{Problema}
\newtheorem{construction}[thm]{Construcción}
\newtheorem{algorithm}[thm]{Algoritmo}
\newtheorem{obs}[thm]{Observación}
\newtheorem{observations}[thm]{Observaciones}
%\renewcommand{\theremark}{}
\newtheorem{note}[thm]{Nota}            %\renewcommand{\thenote}{}
\newtheorem{summary}[thm]{Resumen}         %\renewcommand{\thesumm}{}
\newtheorem{acknowledgement}{Agradecimientos}       \renewcommand{\theack}{}
\newtheorem{notation}[thm]{Notación}
\newtheorem{atention}[thm]{Atención}
\newtheorem{definition-theorem}[thm]{Definición-Teorema}
\newtheorem{definition-lemma}[thm]{Definición-Lema}
\newtheorem{convention}[thm]{Convención}
\newtheorem{application}[thm]{Aplicación}



\theoremstyle{remark}
\newtheorem*{claim}{Claim}
%\newtheorem*{rem}{Remark}
%\newtheorem*{rems}{Remarks}
\newtheorem{case}{Caso}
\newtheorem{subcase}{Subcaso}
\newtheorem{step}{Paso}
\newtheorem{approach}{Enfoque}
%\newtheorem{principle}{Principle}
\newtheorem{fact}[thm]{Hecho}
\newtheorem{subsay}{}
\newtheorem*{notation-and-definition}{Notación y definición}
\newtheorem{assumption}[thm]{Hipótesis}
\newtheorem{remark}[thm]{Observación}
\newtheorem{remarks}[thm]{Observaciones}
\newtheorem{example}[thm]{Ejemplo}


\numberwithin{equation}{section}

\def\factor#1.#2.{\left. \raise 2pt\hbox{$#1$} \right/\hskip -2pt\raise -2pt\hbox{$#2$}}
































%%%%%%%%%%%%%%%%%%%%%%%%%%%%%%%%%%%%%%%%%%%%%%%%%%%%%%%%%%%%%%%%%%%%%%%


%%%%%%%%%%%%%Teoría de Grupos%%%%%%%%%%%%

%Grupo simétrico de n elementos
\newcommand{\SymGrp}[1]{\mathbb{S}_{#1}}
%Grupo alternado de n elementos
\newcommand{\AltGrp}[1]{\mathbb{A}_{#1}}

%Orden de un elemento $a \in G$ de un grupo
\newcommand{\ord}[1]{\operatorname{ord} (#1)}







%%%%%%%%%%%%%%%Polinomios%%%%%%%%%%%%%%%%%%%%


%grado de una extensión algebraica
\newcommand{\degExt}[2]{[#1:#2]}
\newcommand{\degSep}[2]{[#1 : #2]_s}
\newcommand{\degInsep}[2]{[#1:#2]_i}


%espacio afin A^n
\newcommand{\afine}[1]{\mathbb{A}^{#1}}
%espacio proyectivo P^n
\newcommand{\projective}[1]{\mathbb{P}^{#1}}







%%%%%%%%%%%%%%%%%%%%%%%%%%%%%%%%%%%


%grupos de matrices
%SL
\newcommand{\SL}[2]{\operatorname{SL}_{#1} ( #2)}
%GL
\newcommand{\GL}[2]{\operatorname{GL}_{#1} ( #2)}

%matriz identidad
\newcommand{\Id}{\operatorname{Id}}



%enteros Z
\newcommand{\integers}{\mathbb{Z}}
%racionales
\newcommand{\rationals}{\mathbb{Q}}
%naturales
\newcommand{\naturals}{\mathbb{N}}
%reales R
\newcommand{\reals}{\mathbb{R}}
%imaginarios
\newcommand{\complex}{\mathbb{C}}
%p-adicos
\newcommand{\padics}{\mathbb{Q}_p}
%enteros p-adicos
\newcommand{\padicintegers}{\mathbb{Z}_p}

%cuerpos finitos
%Fp
\newcommand{\Fp}{\mathbb{F}_p}
%Fq
\newcommand{\Fq}{\mathbb{F}_q}



%valor absoluto
\newcommand{\abs}[1]{\left \vert #1 \right \vert}
%valor absoluto con dos barras
\newcommand{\Abs}[1]{\left \vert \left \vert #1 \right \vert \right \vert}

%valuacion p-adica
\newcommand{\val}[1]{\operatorname{val} (#1)}

%Hom
\newcommand{\Hom}[3]{\operatorname{Hom}_{#1} (#2, #3)}
%Hom con caligrafia cursiva
\newcommand{\HomCalli}[3]{\operatorname{\text{\calligra{Hom}}}_{\: \: \: #1} (#2, #3)}

%imagen y núcleo
\newcommand{\Imagen}{\operatorname{Im}}
\newcommand{\Ker}{\operatorname{Ker}}

%coker
\newcommand{\Coker}{\operatorname{Coker}}

%limite inverso
\newcommand{\liminv}{\varprojlim}


%un poco de typeset para categorias
\newcommand{\catname}[1]{{\operatorfont\textbf{#1}}}

%flecha de isomorfismo a derecha corto \isomrightarrow
\newcommand{\isomrightarrow}{\overset{\sim}{\rightarrow}}
%flecha de isomorfismo a derecha largo \isomrightarrow
\newcommand{\isomlongrightarrow}{\overset{\sim}{\longrightarrow}}

%flecha de isomorfismo a izquierda corto \isomleftarrow
\newcommand{\isomleftarrow}{\overset{\sim}{\leftarrow}}
%flecha de isomorfismo a izquierda largo \isomleftarrow
\newcommand{\isomlongleftarrow}{\overset{\sim}{\longleftarrow}}


\renewcommand{\hat}[1]{\widehat{#1}}
\renewcommand{\bar}[1]{\overline{#1}}
\renewcommand{\tilde}[1]{\widetilde{#1}}

%declaro un comando nuevo para escribir restricción de funciones
\newcommand\rest[2]{{% we make the whole thing an ordinary symbol
  \left.\kern-\nulldelimiterspace % automatically resize the bar with \right
  #1 % the function
  \vphantom{\big|} % pretend it's a little taller at normal size
  \right|_{#2} % this is the delimiter
  }}


%%%%   COMANDO ALGEBRA CONMUTATIVA   %%%%

%altura de un ideal:
\newcommand{\height}{\textsc{height}}

%Clausura topológica
\newcommand{\closure}[1]{\overline{#1}}

%longitud de un A-modulo. Notacion: \length_A M
\newcommand{\length}{\operatorname{length}}

%Anulador de un $A$-módulo.
\newcommand{\Ann}[1]{\operatorname{Ann} (#1)}

%Cuerpo de fracciones. Notacion $\FracField A$.
\newcommand{\FracField}[1]{\operatorname{Fr} (#1)}

\newcommand{\Spec}[1]{\operatorname{Spec}(#1)}

%conjunto de Lugares de un cuerpo
\newcommand{\places}[1]{\mathcal{Pl} (#1)}

%volumen
\newcommand{\Vol}[1]{\operatorname{vol}\left ( #1 \right)}


%%%%%%%%%%%%%%%%%%%%%%%%%%%%%%%%%%%%



%%%%   COMANDO ANÁLISIS  %%%%

%definimos el diferencial d de la integral "\int f(x) \dd x"
\newcommand*\dd{\mathop{}\!\mathrm{d}}

%definimos mas diferenciales
\newcommand{\dmu}[1]{\dd \mu (#1)}
\newcommand{\dnu}[1]{\dd \nu (#1)}
\newcommand{\dtheta}[1]{\dd \theta (#1)}
\newcommand{\dxi}[1]{\dd \xi (#1)}
\newcommand{\deta}[1]{\dd \eta (#1)}






%%%%   COMANDO TEORÍA DE NÚMEROS  %%%%

%Morfismo de Frobenius
\newcommand{\Frob}{\operatorname{Frob}}

%Grupo de Galois
\newcommand{\Gal}[2]{\operatorname{Gal} ( #1 / #2 )}

%Discriminante
\newcommand{\discriminant}[1]{\mathfrak{d} (#1 )}
\newcommand{\disc}{\operatorname{d}}
\newcommand{\Disc}[3]{\operatorname{D}_{#1 / #2} (#3)}

%%%%Ideales primos%%%
%escribe una letra en notación mathfrak, para denotar a un ideal o elemento primo.

\newcommand{\primo}[1]{\mathfrak{#1}}
\newcommand{\Primo}[1]{\mathfrak{\MakeUppercase{#1}}}

%anillo de enteros O_K
\renewcommand{\O}{\mathcal{O}}
%anillo de enteros con subindice de cuerpo (input, por ejemplo $K$).
\newcommand{\integralring}[1]{O_{#1}}

%caracteristica de un cuerpo Char k
\newcommand{\Char}[1]{\operatorname{Char} #1}

%traza. Notación \trace = Tr
\newcommand{\trace}{\operatorname{Tr}}

%Traza de extensiones. Notación \Tr L K \alpha = \operatorname{Tr}_{L/K} (\alpha)
\newcommand{\Tr}[1]{\operatorname{Tr}_{L/K} (#1)} %la extension es L/K por default
\newcommand{\tr}[3]{\operatorname{Tr}_{#1/#2} (#3)}

%Norma de extensiones. Notación \Norm L K \alpha = \operatorname{N}_{L/K} (\alpha)
\newcommand{\Norm}[1]{\operatorname{N}_{L/K} (#1)}%la extension es L/K por default
\newcommand{\norm}[3]{\operatorname{N}_{#1/#2} (#3)}

%%%%%%%%%%%%%%%%%%%%%%%%%%%%%%%%%%%%

%COMANDOS GEOMETRIA ALGEBRAICA

%grupo de cohomología H^n (U,V)
\renewcommand{\H}[3]{\operatorname{H}^{#1} (#2, #3)}










\begin{document}

\title{Teorema del punto fijo de Borel}

\author{Enzo \textsc{Giannotta}}

%\address{}

%\email{}



\begin{abstract}
En este artículo probaremos el Teorema del punto fijo de Borel, el cual dice que todo grupo algebraico afín y soluble actuando en una variedad algebraica proyectiva admite un punto fijo.
\end{abstract}

\maketitle

\tableofcontents
%{\small\tableofcontents}

\epigraph{I feel that what mathematics needs least are pundits who issue prescriptions or guidelines for presumably less enlightened mortals.}{\textit{Armand Borel}}

\section{Introducción}


La verdadera geometría algebraica comienza al considerar ecuaciones polinomiales cúbicas. Todo aquello de grado menor, tales como aplicaciones lineales o formas cuadráticas, puede ser pensado utilizando métodos de álgebra lineal. Una gran cantidad de trabajo, desde los comienzos de la geometría algebraica hasta nuestros días, ha sido dedicado al estudio de ecuaciones cúbicas. Por ejemplo, las hipersuperficies cúbicas de dimensión 1 son llamadas \emph{curvas elípticas} y ocupan un lugar central en geometría algebraica y aritmética.

\vspace{2mm}

El propósito de este artículo es estudiar superficies cúbicas, es decir, superficies $S\subseteq \mathbb{P}^3(k)$ dadas por un polinomio homogéneo $f(x_0,x_1,x_2,x_3)$ de grado 3. Más precisamente, probaremos el siguiente resultado descubierto originalmente por Cayley y Salmon en 1849.

\begin{thma}\label{teo:Teorema A}
Sea $k$ un cuerpo algebraicamente cerrado. Entonces, toda superficie cúbica suave $S\subseteq \mathbb{P}^3(k)$ posee exactamente 27 rectas.
\end{thma}

Recordemos que una variedad algebraica $X$ es \emph{racional} si posee un abierto de Zariski no-vacío $U\subseteq X$ isomorfo a un abierto no-vacío $V$ del espacio afín $\mathbb{A}^n(k)$. Como aplicación del teorema anterior, probaremos que toda superficie cúbica suave es racional. Más precisamente, probaremos el siguiente resultado.

\begin{thmb}\label{teo:Teorema B}
Sea $k$ un cuerpo algebraicamente cerrado. Entonces, toda superficie cúbica suave $S\subseteq \mathbb{P}^3(k)$ es isomorfa al blow-up del plano proyectivo $\mathbb{P}^2(k)$ en 6 puntos.
\end{thmb}

\subsection*{Estructura del artículo}
La Sección 2 recopila notaciones, convenciones y hechos conocidos que serán usados a lo largo del artículo. También estableceremos algunos hechos preliminares. En particular, discutimos el hecho que el espacio de parámetros de superficies cúbicas en $\mathbb{P}^3$ es isomorfo a $\mathbb{P}^{19}$, y que las superficies singulares forman un divisor irreducible (i.e., una hipersuperficie de dimensión 18 dentro de dicho $\mathbb{P}^{19}$). La Sección 3 está dedicada a probar el Teorema A. Finalmente, en la Sección 4 recordamos el concepto de racionalidad y probamos el Teorema B.

%%% Opcional (generalmente se agradece a gente con la quien se discutió o preguntó)
\subsection*{Agradecimientos}

Agradezco al profesor Pedro por sugerir este tema para el artículo, y por hacer disponible el material bibliográfico necesario para prepararlo.

\section{Notación, convenciones y hechos preliminares}

\subsection{Convención} Durante todo el artículo, todas las variedades y morfismos estarán definidos sobre un cuerpo $k$ algebraicamente cerrado.

\subsection{Notación}
Denotamos por $\mathbb{P}^n$ (resp. $\mathbb{A}^n$) al espacio proyectivo $\mathbb{P}^n(k)$ (resp. espacio afín $\mathbb{A}^n(k)$) de dimensión $n$ sobre el cuerpo $k$.

Dada una variedad $X$, denotamos por $X_{\textup{sing}}$ al sub-conjunto de puntos singulares de $X$. En particular, $X$ es una variedad suave si y sólo si $X_{\textup{sing}}=\emptyset$.

\subsection{Hechos preliminares}

Comencemos por definir qué entenderemos por una superficie cúbica.

\begin{defn}\label{defn:sup cubica}
Sea $f(x_0,x_1,x_2,x_3)\in k[x_0,x_1,x_2,x_3]$ un polinomio homogéneo de grado $3$ no-nulo. Diremos que la variedad proyectiva
$$S=\{[x_0,x_1,x_2,x_3]\in \mathbb{P}^3\;|\;f(x_0,x_1,x_2,x_3)=0\}\subseteq \mathbb{P}^3 $$
es una {\bf superficie cúbica}.
\end{defn}

A continuación recordamos la definición de una recta en un espacio proyectivo.

\begin{defn}\label{defn:recta}
Sea $\mathbb{P}^n$ el espacio proyectivo asociado al $k$-espacio vectorial $V=k^{n+1}$. Por definición, una {\bf recta} $\ell$ en $\mathbb{P}^n$ es un sub-espacio proyectivo $\ell \cong \mathbb{P}^1$ de $\mathbb{P}^n$ asociado a un sub-espacio vectorial $W\cong k^2$.
\end{defn}

\begin{exmp}\label{exmp:Fermat}
Sea $k=\mathbb{C}$ el cuerpo de los números complejos. La superficie cúbica de Fermat en $\mathbb{P}^3$ dada por la ecuación
$$x_0^3+x_1^3+x_2^3+x_3^3=0$$
posee exactamente 27 rectas, cada una de la forma
$$\ell_{(i,j,k)} := \{[x_0,x_1,x_2,x_3]\in \mathbb{P}^3\;|\;x_0+\omega x_i = x_j+\omega'x_k=0\} \cong \mathbb{P}^1, $$
donde $\{i,j,k\}=\{1,2,3\}$, $j<k$, y donde $\omega$ y $\omega'$ son raíces cúbicas de la unidad (eventualmente la misma).

\vspace{2mm}

En efecto, (...).
\end{exmp}


\begin{lemma}
Sea $\mathcal{S}$ el espacio de parámetros de superficies cúbicas en $\mathbb{P}^3$. Entonces, $\mathcal{S}\cong \mathbb{P}^{19}$.
\end{lemma}

\begin{proof}
El espacio vectorial de polinomios homogéneos cúbicos en 4 variables puede ser identificado con el espacio de secciones globales $\textup{H}^0(\mathbb{P}^3,\mathcal{O}_{\mathbb{P}^3}(3))$. Más aún, tenemos que
$$\dim_k \textup{H}^0(\mathbb{P}^n,\mathcal{O}_{\mathbb{P}^n}(d))=\binom{n+d}{d}, $$
por lo que en nuestro caso $\dim_k \textup{H}^0(\mathbb{P}^3,\mathcal{O}_{\mathbb{P}^3}(3)) = \binom{6}{3}=20$. Finalmente, notamos que dos polinomios proporcionales definen la misma superficie cúbica y que, por definición, no debemos considerar el polinomio nulo. En otras palabras, el espacio de parámetros $\mathcal{S}$ está dado por la proyectivización $\mathcal{S}=\mathbb{P}(\textup{H}^0(\mathbb{P}^3,\mathcal{O}_{\mathbb{P}^3}(3)))\cong \mathbb{P}^{19}$.
\end{proof}

El siguiente hecho será utilizado directamente sin demostración (ver \cite[\S 7.1]{3264} para más detalles) en el caso donde el cuerpo $k$ es de característica nula\footnote{La demostración pasa por el teorema de Bertini, que sólo es válido en $\textup{car}(k)=0$.}.

\begin{fact}
Supongamos que $\textup{car}(k)=0$. Entonces, el conjunto de superficies cúbicas singulares corresponde a una hipersuperficie irreducible $\sD$ dentro del espacio de parámetros $\mathcal{S}$, llamada usualmente el {\bf divisor discriminante}.
\end{fact}

\section{Teorema del punto fijo de Borel}

En esta sección probaremos el teorema principal de este artículo:



\section{Aplicación: toda superficie cúbica suave es racional}

En esta sección probaremos el Teorema B, lo cual implicará en particular que toda superficie cúbica suave (sobre un cuerpo algebraicamente cerrado) es racional. Comencemos por recordar la noción de racionalidad, recordar la noción de blow-up, y notar que todo blow-up del espacio proyectivo $\mathbb{P}^n$ es racional.








\section{Cosas para agregar}
Notacion: B_u, B_s$

definiciones basicas de grupos: solubilidad, subgrupo normal,  nilpotencia, toro maximal, unipotente conexo maximal que es una bandera, que es una bandera completa, una bandera standard (creo que es la bandera de $V = k^n$). Que es un grupo de isotropia. Que es una orbita. Que es el estabilizador. Que es una clase de Conjugación.

definiciones algebraicas: que es un grupo algebraico, que es un grupo algebraico conexo. que es la componente conexa $G^\circ$. Como es la estructura algebraica de un espacio homogeneo. Que es una accion de un grupo algebraico $G$ en una variedad algebraica X. que es la variedad bandera de un espacio vectorial finito $V$, dar como ejemplo la accion natural de $\operatorname{GL}(V)$ allí. Que es un morfismo de grupos algebraicos.

Teoremas de Humphreys: Teorema 11.2, 11.3, (8.3), Teorema 19.3, 19.2, Proposición 17.4, Corolario 16.3. Proposición 7.4B. No copiar solo citar el Teorema 22.4, el 22.3 y el 22.2.

Hacer una observación para no volver a repetirlo que todo grupo algebraico de dimensión $0$ tiene que ser el grupo trivial $\{1\}$.

Ejercicios: 21.5, 21.2





\bibliographystyle{amsalpha}
\bibliography{main}



\end{document}
