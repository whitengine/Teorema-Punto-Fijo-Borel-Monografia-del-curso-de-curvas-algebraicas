%\documentclass[english,10pt,draft]{amsart}
%\documentclass[english,10pt]{amsart}

\documentclass[spanish,10pt]{amsart}

%\usepackage[english]{babel}
\usepackage[spanish]{babel}

\relpenalty=9999
\binoppenalty=9999
%\mathsurround=1pt %espacio antes y después de una fórmula en el texto


%\usepackage[matrix,arrow]{xy}
%\xyoption{all}

\usepackage{amscd,amssymb,amsfonts,amsmath}
\usepackage{tikz-cd}

\usepackage{bm}
\usepackage{graphics}
\usepackage{epsfig}
\usepackage{mathrsfs}
\usepackage{xcolor}
\DeclareMathAlphabet{\mathscrbf}{OMS}{mdugm}{b}{n}

\usepackage{epigraph}

\usepackage{braket} %para definir \set , \Set y que los conjuntos se vean mas lindos

\usepackage[shortlabels]{enumitem}

\definecolor{violet}{rgb}{0.0,0.2,0.7}
\definecolor{rouge2}{rgb}{0.8,0.0,0.2}
\usepackage{hyperref}
\hypersetup{
    %bookmarks=true,         % show bookmarks bar?
    unicode=false,          % non-Latin characters in Acrobat’s bookmarks
    pdftoolbar=true,        % show Acrobat’s toolbar?
    pdfmenubar=true,        % show Acrobat’s menu?
    pdffitwindow=false,     % window fit to page when opened
    pdfstartview={FitH},    % fits the width of the page to the window
    pdftitle={},    % title
    pdfauthor={},     % author
    colorlinks=true,       % false: boxed links; true: colored links
   linkcolor=violet,          % color of internal links
    citecolor=rouge2,        % color of links to bibliography
    filecolor=black,      % color of file links
    urlcolor=cyan}           % color of external links


%\renewcommand{\thepart}{\Roman{part}}
\setcounter{tocdepth}{1} %tableofcontents

\unitlength=1cm

\usepackage[text={6.7in,9.2in},centering]{geometry}

\makeatletter
\renewcommand\subsection{\@startsection{subsection}{2}%
  \z@{.5\linespacing\@plus.7\linespacing}{-.5em}%
  %{\normalfont\scshape}}
  %{\normalfont\itshape}}
  {\normalfont\sffamily}}
\makeatother

%\renewcommand{\baselinestretch}{1.1}
%\renewcommand{\baselinestretch}{1.2}


\newcommand{\cqfd}{\hfill $\square$}
\newcommand{\ie}{\textit{i.e. }}

\newcommand{\Hilb}{\textup{Hilb}}
\newcommand{\Pic}{\textup{Pic}}
\newcommand{\Alb}{\textup{Alb}}
\newcommand{\Spec}{\textup{Spec}}
\newcommand{\Proj}{\textup{Proj}}
\newcommand{\Supp}{\textup{Supp}}
\newcommand{\Exc}{\textup{Exc}}
\newcommand{\Rat}{\textup{RatCurves}}
\newcommand{\RatCurves}{\textup{RatCurves}}
\newcommand{\RC}{\textup{RatCurves}^n}
\newcommand{\Univ}{\textup{Univ}}
\newcommand{\Chow}{\textup{Chow}}
\newcommand{\Sing}{\textup{Sing}}
\newcommand{\CS}{\textup{CS}}
\newcommand{\codim}{\textup{codim}}
\newcommand{\Res}{\textup{Res}}
\newcommand{\modulo}{\textup{mod}}




%\let \cedilla =\c
%\renewcommand{\c}[0]{{\mathbb C}}
%\newcommand{\p}[0]{{\mathbb P}}


\newcommand{\Nef}{\textup{Nef}}
%\renewcommand{\Big}{\textup{Big}}
\renewcommand{\div}{\textup{div}}
\newcommand{\Div}{\textup{Div}}
\newcommand{\Amp}{\textup{Amp}}
\newcommand{\Pef}{\textup{Pef}}
\newcommand{\Eff}{\textup{Eff}}
\newcommand{\NS}{\textup{N}^1}
\newcommand{\N}{\textup{N}}
\newcommand{\Z}{\textup{Z}}
\renewcommand{\H}{\textup{H}}
\newcommand{\h}{\textup{h}}
\newcommand{\discrep}{\textup{discrep}}

\newcommand{\NE}{\overline{\textup{NE}}}
\newcommand{\Hol}{\textup{Hol}}

\newcommand{\into}{\hookrightarrow}
\newcommand{\map}{\dashrightarrow}
\newcommand{\lra}{\longrightarrow}

\renewcommand{\le}{\leqslant}
\renewcommand{\ge}{\geqslant}
\newcommand{\Diff}{\textup{Diff}}
\newcommand{\D}{\Delta}
\renewcommand{\d}{\delta}
\newcommand{\G}{\Gamma}
\newcommand{\Fix}{\textup{Fix}}
\newcommand{\Bs}{\textup{Bs}}
\newcommand{\mult}{\textup{mult}}
\newcommand{\Mob}{\textup{Mob}}
\newcommand{\B}{\textup{B}}

\newcommand{\bA}{\textbf{A}}
\newcommand{\bB}{\textup{\textbf{A}}}
\newcommand{\bC}{\textup{\textbf{C}}}
\newcommand{\bD}{\textbf{D}}
\newcommand{\bE}{\textbf{E}}
\newcommand{\bF}{\textbf{F}}
\newcommand{\bG}{\textbf{G}}
\newcommand{\bH}{\textbf{H}}
\newcommand{\bI}{\textbf{I}}
\newcommand{\bJ}{\textbf{J}}
\newcommand{\bK}{\textbf{K}}
\newcommand{\bL}{\textbf{L}}
\newcommand{\bM}{\textbf{M}}
\newcommand{\bN}{\textbf{N}}
\newcommand{\bO}{\textbf{O}}

%\newcommand{\bP}{\mathbb{P}}
\newcommand{\bQ}{\mathbb{Q}}
%\newcommand{\bZ}{\mathbb{Z}}

\newcommand{\bR}{\textup{\textbf{R}}}
\newcommand{\bS}{\textbf{S}}
\newcommand{\bT}{\textbf{T}}
\newcommand{\bU}{\textbf{U}}
\newcommand{\bV}{\textbf{V}}
\newcommand{\bW}{\textbf{W}}
\newcommand{\bX}{\textbf{X}}
\newcommand{\bY}{\textbf{Y}}
\newcommand{\bZ}{\textbf{Z}}
\renewcommand{\bB}{\textbf{B}}
\newcommand{\bP}{\textbf{P}}


\newcommand{\cA}{\mathcal{A}}
\newcommand{\cB}{\mathcal{B}}
\newcommand{\cC}{\mathcal{C}}
\newcommand{\cD}{\mathcal{D}}
\newcommand{\cE}{\mathcal{E}}
\newcommand{\cF}{\mathcal{F}}
\newcommand{\cG}{\mathcal{G}}
\newcommand{\cH}{\mathcal{H}}
\newcommand{\cI}{\mathcal{I}}
\newcommand{\cJ}{\mathcal{J}}
\newcommand{\cK}{\mathcal{K}}
\newcommand{\cL}{\mathcal{L}}
\newcommand{\cM}{\mathcal{M}}
\newcommand{\cN}{\mathcal{N}}
\newcommand{\cO}{\mathcal{O}}
\newcommand{\cP}{\mathcal{P}}
\newcommand{\cQ}{\mathcal{Q}}
\newcommand{\cR}{\mathcal{R}}
\newcommand{\cS}{\mathcal{S}}
\newcommand{\cT}{\mathcal{T}}
\newcommand{\cU}{\mathcal{U}}
\newcommand{\cV}{\mathcal{V}}
\newcommand{\cW}{\mathcal{W}}
\newcommand{\cX}{\mathcal{X}}
\newcommand{\cY}{\mathcal{Y}}
\newcommand{\cZ}{\mathcal{Z}}

\newcommand{\frA}{\mathfrak{A}}
\newcommand{\frB}{\mathfrak{B}}
\newcommand{\frC}{\mathfrak{C}}
\newcommand{\frD}{\mathfrak{D}}
\newcommand{\frE}{\mathfrak{E}}
\newcommand{\frF}{\mathfrak{F}}
\newcommand{\frG}{\mathfrak{G}}
\newcommand{\frH}{\mathfrak{H}}
\newcommand{\frI}{\mathfrak{I}}
\newcommand{\frJ}{\mathfrak{J}}
\newcommand{\frK}{\mathfrak{K}}
\newcommand{\frL}{\mathfrak{L}}
\newcommand{\frM}{\mathfrak{M}}
\newcommand{\frN}{\mathfrak{N}}
\newcommand{\frO}{\mathfrak{O}}
\newcommand{\frP}{\mathfrak{P}}
\newcommand{\frQ}{\mathfrak{Q}}
\newcommand{\frR}{\mathfrak{R}}
\newcommand{\frS}{\mathfrak{S}}
\newcommand{\frT}{\mathfrak{T}}
\newcommand{\frU}{\mathfrak{U}}
\newcommand{\frV}{\mathfrak{V}}
\newcommand{\frW}{\mathfrak{W}}
\newcommand{\frX}{\mathfrak{X}}
\newcommand{\frY}{\mathfrak{Y}}

%\renewcommand{\frm}{\mathfrak{m}} % accolades dans xymatrix

\newcommand{\frm}{\mathfrak{m}}
\newcommand{\frf}{\mathfrak{f}}
\newcommand{\frg}{\mathfrak{g}}
\newcommand{\frt}{\mathfrak{t}}


\newcommand{\sA}{\mathscr{A}}
\newcommand{\sB}{\mathscr{B}}
\newcommand{\sC}{\mathscr{C}}
\newcommand{\sD}{\mathscr{D}}
\newcommand{\sE}{\mathscr{E}}
\newcommand{\sF}{\mathscr{F}}
\newcommand{\sG}{\mathscr{G}}
\newcommand{\sH}{\mathscr{H}}
\newcommand{\sI}{\mathscr{I}}
\newcommand{\sJ}{\mathscr{J}}
\newcommand{\sK}{\mathscr{K}}
\newcommand{\sL}{\mathscr{L}}
\newcommand{\sM}{\mathscr{M}}
\newcommand{\sN}{\mathscr{N}}
\newcommand{\sO}{\mathscr{O}}
\newcommand{\sP}{\mathscr{P}}
\newcommand{\sQ}{\mathscr{Q}}
\newcommand{\sR}{\mathscr{R}}
\newcommand{\sS}{\mathscr{S}}
\newcommand{\sT}{\mathscr{T}}
\newcommand{\sU}{\mathscr{U}}
\newcommand{\sV}{\mathscr{V}}
\newcommand{\sW}{\mathscr{W}}
\newcommand{\sX}{\mathscr{X}}
\newcommand{\sY}{\mathscr{Y}}
\newcommand{\sZ}{\mathscr{Z}}

\newcommand{\sbfE}{\mathscrbf{E}}
\newcommand{\sbfG}{\mathscrbf{G}}
\newcommand{\sbfH}{\mathscrbf{H}}
\newcommand{\sbfL}{\mathscrbf{L}}
\newcommand{\sbfN}{\mathscrbf{N}}

\newcommand{\art}{\textup{Art}}
\newcommand{\ens}{\textup{Ens}}
\newcommand{\sch}{\textup{Sch}}
\newcommand{\Der}{\textup{Der}}



\newtheorem{theorem}{Teorema}[section]
\newtheorem*{theorema}{Teorema A}
\newtheorem*{theoremb}{Teorema B}
\newtheorem*{theoremc}{Teorema C}
\newtheorem*{theoremd}{Teorema D}
\newtheorem*{theoreme}{Teorema E}
\newtheorem{maintheorem}[theorem]{Teorema Principal}
\newtheorem{question}[theorem]{Pregunta}
\newtheorem{lemma}[theorem]{Lema}
\newtheorem{corollary}[theorem]{Corolario}
\newtheorem{corollaries}[theorem]{Corolarios}
\newtheorem{proposition}[theorem]{Proposición}
\newtheorem{criteria}[theorem]{Criterio}
\newtheorem{conjecture}[theorem]{Conjectura}
\newtheorem{principle}[theorem]{Principio}
\newtheorem{warning}[theorem]{Cuidado}

\newtheorem*{theorem*}{Teorema}
\theoremstyle{definition}
\newtheorem{definition}[theorem]{Definición}
\newtheorem{condition}[theorem]{Condición}
\newtheorem{say}[theorem]{}
\newtheorem{hint}[theorem]{Hint}
\newtheorem{trick}[theorem]{Truco}
\newtheorem{exercise}[theorem]{Ejercicio}
\newtheorem{problem}[theorem]{Problema}
\newtheorem{construction}[theorem]{Construcción}
\newtheorem{algorithm}[theorem]{Algoritmo}
\newtheorem{obs}[theorem]{Observación}
\newtheorem{observations}[theorem]{Observaciones}
%\renewcommand{\theremark}{}
\newtheorem{note}[theorem]{Nota}            %\renewcommand{\thenote}{}
\newtheorem{summary}[theorem]{Resumen}         %\renewcommand{\thesumm}{}
\newtheorem{acknowledgement}{Agradecimientos}
\newtheorem{notation}[theorem]{Notación}
\newtheorem{atention}[theorem]{Atención}
\newtheorem{definition-theorem}[theorem]{Definición-Teorema}
\newtheorem{definition-lemma}[theorem]{Definición-Lema}
\newtheorem{convention}[theorem]{Convención}
\newtheorem{application}[theorem]{Aplicación}



\theoremstyle{remark}
\newtheorem*{claim}{Claim}
%\newtheorem*{rem}{Remark}
%\newtheorem*{rems}{Remarks}
\newtheorem{case}{Caso}
\newtheorem{subcase}{Subcaso}
\newtheorem{step}{Paso}
\newtheorem{approach}{Enfoque}
%\newtheorem{principle}{Principle}
\newtheorem{fact}[theorem]{Hecho}
\newtheorem{subsay}{}
\newtheorem*{notation-and-definition}{Notación y definición}
\newtheorem{assumption}[theorem]{Hipótesis}
\newtheorem{remark}[theorem]{Observación}
\newtheorem{remarks}[theorem]{Observaciones}
\newtheorem{example}[theorem]{Ejemplo}


\numberwithin{equation}{section}

\def\factor#1.#2.{\left. \raise 2pt\hbox{$#1$} \right/\hskip -2pt\raise -2pt\hbox{$#2$}}
































%%%%%%%%%%%%%%%%%%%%%%%%%%%%%%%%%%%%%%%%%%%%%%%%%%%%%%%%%%%%%%%%%%%%%%%


%%%%%%%%%%%%%Teoría de Grupos%%%%%%%%%%%%

%Grupo simétrico de n elementos
\newcommand{\SymGrp}[1]{\mathbb{S}_{#1}}
%Grupo alternado de n elementos
\newcommand{\AltGrp}[1]{\mathbb{A}_{#1}}

%Orden de un elemento $a \in G$ de un grupo
\newcommand{\ord}[1]{\operatorname{ord} (#1)}







%%%%%%%%%%%%%%%Polinomios%%%%%%%%%%%%%%%%%%%%


%grado de una extensión algebraica
\newcommand{\degExt}[2]{[#1:#2]}
\newcommand{\degSep}[2]{[#1 : #2]_s}
\newcommand{\degInsep}[2]{[#1:#2]_i}


%espacio afin A^n
\newcommand{\afine}[1]{\mathbb{A}^{#1}}
%espacio proyectivo P^n
\newcommand{\projective}[1]{\mathbb{P}^{#1}}







%%%%%%%%%%%%%%%%%%%%%%%%%%%%%%%%%%%


%grupos de matrices
%SL
\newcommand{\SL}[2]{\operatorname{SL}_{#1} ( #2)}
%GL
\newcommand{\GL}[2]{\operatorname{GL}_{#1} ( #2)}

%matriz identidad
\newcommand{\Id}{\operatorname{Id}}



%enteros Z
\newcommand{\integers}{\mathbb{Z}}
%racionales
\newcommand{\rationals}{\mathbb{Q}}
%naturales
\newcommand{\naturals}{\mathbb{N}}
%reales R
\newcommand{\reals}{\mathbb{R}}
%imaginarios
\newcommand{\complex}{\mathbb{C}}
%p-adicos
\newcommand{\padics}{\mathbb{Q}_p}
%enteros p-adicos
\newcommand{\padicintegers}{\mathbb{Z}_p}

%cuerpos finitos
%Fp
\newcommand{\Fp}{\mathbb{F}_p}
%Fq
\newcommand{\Fq}{\mathbb{F}_q}



%valor absoluto
\newcommand{\abs}[1]{\left \vert #1 \right \vert}
%valor absoluto con dos barras
\newcommand{\Abs}[1]{\left \vert \left \vert #1 \right \vert \right \vert}

%valuacion p-adica
\newcommand{\val}[1]{\operatorname{val} (#1)}

%Hom
\newcommand{\Hom}[3]{\operatorname{Hom}_{#1} (#2, #3)}
%Hom con caligrafia cursiva
\newcommand{\HomCalli}[3]{\operatorname{\text{\calligra{Hom}}}_{\: \: \: #1} (#2, #3)}

%imagen y núcleo
\newcommand{\Imagen}{\operatorname{Im}}
\newcommand{\Ker}{\operatorname{Ker}}

%coker
\newcommand{\Coker}{\operatorname{Coker}}

%limite inverso
\newcommand{\liminv}{\varprojlim}


%un poco de typeset para categorias
\newcommand{\catname}[1]{{\operatorfont\textbf{#1}}}

%flecha de isomorfismo a derecha corto \isomrightarrow
\newcommand{\isomrightarrow}{\overset{\sim}{\rightarrow}}
%flecha de isomorfismo a derecha largo \isomrightarrow
\newcommand{\isomlongrightarrow}{\overset{\sim}{\longrightarrow}}

%flecha de isomorfismo a izquierda corto \isomleftarrow
\newcommand{\isomleftarrow}{\overset{\sim}{\leftarrow}}
%flecha de isomorfismo a izquierda largo \isomleftarrow
\newcommand{\isomlongleftarrow}{\overset{\sim}{\longleftarrow}}


\renewcommand{\hat}[1]{\widehat{#1}}
\renewcommand{\bar}[1]{\overline{#1}}
\renewcommand{\tilde}[1]{\widetilde{#1}}

%declaro un comando nuevo para escribir restricción de funciones
\newcommand\rest[2]{{% we make the whole thing an ordinary symbol
  \left.\kern-\nulldelimiterspace % automatically resize the bar with \right
  #1 % the function
  \vphantom{\big|} % pretend it's a little taller at normal size
  \right|_{#2} % this is the delimiter
  }}


%%%%   COMANDO ALGEBRA CONMUTATIVA   %%%%

%altura de un ideal:
\newcommand{\height}{\textsc{height}}

%Clausura topológica
\newcommand{\closure}[1]{\overline{#1}}

%longitud de un A-modulo. Notacion: \length_A M
\newcommand{\length}{\operatorname{length}}

%Anulador de un $A$-módulo.
\newcommand{\Ann}[1]{\operatorname{Ann} (#1)}

%Cuerpo de fracciones. Notacion $\FracField A$.
\newcommand{\FracField}[1]{\operatorname{Fr} (#1)}


%conjunto de Lugares de un cuerpo
\newcommand{\places}[1]{\mathcal{Pl} (#1)}

%volumen
\newcommand{\Vol}[1]{\operatorname{vol}\left ( #1 \right)}


%%%%%%%%%%%%%%%%%%%%%%%%%%%%%%%%%%%%



%%%%   COMANDO ANÁLISIS  %%%%

%definimos el diferencial d de la integral "\int f(x) \dd x"
\newcommand*\dd{\mathop{}\!\mathrm{d}}

%definimos mas diferenciales
\newcommand{\dmu}[1]{\dd \mu (#1)}
\newcommand{\dnu}[1]{\dd \nu (#1)}
\newcommand{\dtheta}[1]{\dd \theta (#1)}
\newcommand{\dxi}[1]{\dd \xi (#1)}
\newcommand{\deta}[1]{\dd \eta (#1)}






%%%%   COMANDO TEORÍA DE NÚMEROS  %%%%

%Morfismo de Frobenius
\newcommand{\Frob}{\operatorname{Frob}}

%Grupo de Galois
\newcommand{\Gal}[2]{\operatorname{Gal} ( #1 / #2 )}

%Discriminante
\newcommand{\discriminant}[1]{\mathfrak{d} (#1 )}
\newcommand{\disc}{\operatorname{d}}
\newcommand{\Disc}[3]{\operatorname{D}_{#1 / #2} (#3)}

%%%%Ideales primos%%%
%escribe una letra en notación mathfrak, para denotar a un ideal o elemento primo.

\newcommand{\primo}[1]{\mathfrak{#1}}
\newcommand{\Primo}[1]{\mathfrak{\MakeUppercase{#1}}}

%anillo de enteros O_K
\renewcommand{\O}{\mathcal{O}}
%anillo de enteros con subindice de cuerpo (input, por ejemplo $K$).
\newcommand{\integralring}[1]{O_{#1}}

%caracteristica de un cuerpo Char k
\newcommand{\Char}[1]{\operatorname{Char} #1}

%traza. Notación \trace = Tr
\newcommand{\trace}{\operatorname{Tr}}

%Traza de extensiones. Notación \Tr L K \alpha = \operatorname{Tr}_{L/K} (\alpha)
\newcommand{\Tr}[1]{\operatorname{Tr}_{L/K} (#1)} %la extension es L/K por default
\newcommand{\tr}[3]{\operatorname{Tr}_{#1/#2} (#3)}

%Norma de extensiones. Notación \Norm L K \alpha = \operatorname{N}_{L/K} (\alpha)
\newcommand{\Norm}[1]{\operatorname{N}_{L/K} (#1)}%la extension es L/K por default
\newcommand{\norm}[3]{\operatorname{N}_{#1/#2} (#3)}

%%%%%%%%%%%%%%%%%%%%%%%%%%%%%%%%%%%%

%COMANDOS GEOMETRIA ALGEBRAICA

%grupo de cohomología H^n (U,V)
\renewcommand{\H}[3]{\operatorname{H}^{#1} (#2, #3)}










\begin{document}

\title{Teorema del punto fijo de Borel}

\author{Enzo \textsc{Giannotta}}

%\address{}

%\email{}



\begin{abstract}
En este artículo probaremos el Teorema del punto fijo de Borel, el cual dice que todo grupo algebraico afín y soluble actuando en una variedad algebraica proyectiva admite un punto fijo.
\end{abstract}

\maketitle

\tableofcontents
%{\small\tableofcontents}

\epigraph{I feel that what mathematics needs least are pundits who issue prescriptions or guidelines for presumably less enlightened mortals.}{\textit{Armand Borel}}

\section{Introducción}


La verdadera geometría algebraica comienza al considerar ecuaciones polinomiales cúbicas. Todo aquello de grado menor, tales como aplicaciones lineales o formas cuadráticas, puede ser pensado utilizando métodos de álgebra lineal. Una gran cantidad de trabajo, desde los comienzos de la geometría algebraica hasta nuestros días, ha sido dedicado al estudio de ecuaciones cúbicas. Por ejemplo, las hipersuperficies cúbicas de dimensión 1 son llamadas \emph{curvas elípticas} y ocupan un lugar central en geometría algebraica y aritmética.

\vspace{2mm}

El propósito de este artículo es estudiar superficies cúbicas, es decir, superficies $S\subseteq \mathbb{P}^3(k)$ dadas por un polinomio homogéneo $f(x_0,x_1,x_2,x_3)$ de grado 3. Más precisamente, probaremos el siguiente resultado descubierto originalmente por Cayley y Salmon en 1849.

\begin{theorema}\label{teo:Teorema A}
Sea $k$ un cuerpo algebraicamente cerrado. Entonces, toda superficie cúbica suave $S\subseteq \mathbb{P}^3(k)$ posee exactamente 27 rectas.
\end{theorema}

Recordemos que una variedad algebraica $X$ es \emph{racional} si posee un abierto de Zariski no-vacío $U\subseteq X$ isomorfo a un abierto no-vacío $V$ del espacio afín $\mathbb{A}^n(k)$. Como aplicación del teorema anterior, probaremos que toda superficie cúbica suave es racional. Más precisamente, probaremos el siguiente resultado.

\begin{theoremb}\label{teo:Teorema B}
Sea $k$ un cuerpo algebraicamente cerrado. Entonces, toda superficie cúbica suave $S\subseteq \mathbb{P}^3(k)$ es isomorfa al blow-up del plano proyectivo $\mathbb{P}^2(k)$ en 6 puntos.
\end{theoremb}

\subsection*{Estructura del artículo}
La Sección 2 recopila notaciones, convenciones y hechos conocidos que serán usados a lo largo del artículo. También estableceremos algunos hechos preliminares. En particular, discutimos el hecho que el espacio de parámetros de superficies cúbicas en $\mathbb{P}^3$ es isomorfo a $\mathbb{P}^{19}$, y que las superficies singulares forman un divisor irreducible (i.e., una hipersuperficie de dimensión 18 dentro de dicho $\mathbb{P}^{19}$). La Sección 3 está dedicada a probar el Teorema A. Finalmente, en la Sección 4 recordamos el concepto de racionalidad y probamos el Teorema B.

%%% Opcional (generalmente se agradece a gente con la quien se discutió o preguntó)
\subsection*{Agradecimientos}

Agradezco al profesor Pedro por sugerir este tema para el artículo, y por hacer disponible el material bibliográfico necesario para prepararlo.

\section{Notación, convenciones y hechos preliminares}

\subsection{Convención} Durante todo el artículo, todas las variedades y morfismos estarán definidos sobre un cuerpo $k$ algebraicamente cerrado, de característica arbitraria.

\subsection{Notación}
Denotamos por $\mathbb{P}^n$ (resp. $\mathbb{A}^n$)S al \textit{espacio proyectivo} $\mathbb{P}^n(k)$ (resp. \textit{espacio afín} $\mathbb{A}^n(k)$) de dimensión $n$ sobre el cuerpo $k$. Más generalmente, dado un espacio vectorial de dimensión finita $V$, denotamos por $\mathbb{P} (V)$ a la \textit{proyectivización} de $V$, i.e., el conjunto de todos los subespacios vectoriales de dimensión $1$ de $V$. El \textit{grupo multiplicativo} $\mathbb{G}_m := ( k^\times, \cdot)$.

Dada una variedad $X$, denotamos por $X_{\textup{sing}}$ al sub-conjunto de puntos singulares de $X$. En particular, $X$ es una variedad suave si y sólo si $X_{\textup{sing}}=\emptyset$. En contraste, definimos $X_{\textup{reg}}$ al sub-conjunto de puntos suaves de $X$.

\section{Terminología de Teoría de Grupos}
En esta sección introducimos notación clásica de Teoría de Grupos. Sea $G$ un grupo, escribamos
\begin{itemize}
\item $Z(G)$ es el \textbf{centro} de $G$, es decir, todos los $x \in G$ tales que $x g = g x$ para todo $g \in G$.
\item $R(G)$ es el \textbf{radical} de $G$, es decir, el subgrupo conexo normal soluble de $G$. Donde estamos suponiendo que $G$ es un \textit{grupo algebraico}.
\item Dado $S \subset G$, $C_G (S)$ es el \textbf{centralizador} de $S$, es decir, los $x \in G$ tales que $x g = g x$ para todo $g \in S$. Cuando $S = \{x\}$ es un singleton, escribimos $C_G (x)$ en lugar de $C_G (S)$.
\item Dado $S \subset G$, $N_G (S)$ es el \textbf{normalizador} de $S$, es decir, los $x \in G$ tales que $x g x^{-1} \in S$ para todo $g \in S$. Cuando $S = \{x\}$ es un singleton, escribimos $N_G (x)$ en lugar de $N_G (S)$.
\item $[G,G]$ es el \textbf{subgrupo derivado} de $G$, es decir, el subgrupo generado por los conmutadores $[x,y] := x y x^{-1}y^{-1}$ con $x,y \in G$. Más generalmente, si $S_1, S_2 \subset G$, escribimos $[S_1, S_2]$ para el subgrupo generado por los $[s_1,s_2]$ con $s_1 \in S_1 $ y $s_2 \in S_2$.
\item Dado un conjunto arbitrario $X$, una \textbf{acción} de $G$ es un mapa
\begin{align*}
G \times X &\longrightarrow X \\
(g,x) &\longmapsto g \cdot x
\end{align*}
tal que $1 \cdot x = x$ y $(gh) \cdot x = g \cdot (h \cdot x)$ para todo $x \in X$ y $g,h \in G$. Dado un conjunto $S \subset X$, escribimos $G_S$ para el \textbf{estabilizador} de $S$, es decir, los $g \in G$ tales que $g \cdot x = x$ para todo $x \in S$. Cuando $S = \{x\}$ es un singleton, simplemente escribimos $G_x$ en lugar de $G_S$. Por otro lado, $G \cdot S$ es la \textbf{órbita} de $S$, es decir, el conjunto de elementos $x \in X$ tales que $x = g \cdot s$ para algún $g \in G$ y $s \in S$. Cuando $S = \{x\}$ es un singleton, simplemente escribimos $G \cdot x$ en lugar de $G_S$.
\end{itemize}

Introduzcamos también definiciones básicas de Teoría de Grupos. Decimos que un subgrupo

\section{Variedades Bandera}

En esta sección construiremos un ejemplo de variedades proyectivas: las \textit{variedades bandera}.

Sea $V$ un espacio vectorial de dimensión $n$ sobre un cuerpo $k$. $\bigwedge V$ denota el \textbf{álgebra exterior} (el cociente del \textit{álgebra de tensores} de $V$ por el ideal bilátero generado por $v \otimes v$, $v \in V$). Recordemos que $\bigwedge V$ es una $k$-álgebra de dimensión finita, más aún, es una $k$-álgebra graduada por $\{\bigwedge^i V\}_{i = 0}^n$. En particular $\bigwedge^0 V = k$ y $\bigwedge^1 V = V$. Dada una base ordenada $(v_1, \ldots, v_n)$ de $V$, entonces los \textit{productos cuña} $v_{i_1} \wedge \cdots \wedge v_{i_d}$ ($i_1 < i_2 < \cdots < i_d$) forman una $k$-base de $\bigwedge^d V$ de cardinal $\binom n d$. En particular $\bigwedge^n V$ es $1$-dimensional, i.e., el producto cuña de una base arbitraria de $V$ está determinada salvo un múltiplo escalar. Si $W$ es un subespacio vectorial de $V$, entonces podemos identificar canónicamente a $\bigwedge^d W$ con un subespacio de $\bigwedge^d V$.

Así, tenemos un mapa $\psi$ saliendo del conjunto $\mathfrak G_d (V)$ de todos los subespacios $d$-dimensionales de $V$ en $\mathbb{P} (\bigwedge^d V)$, que manda un subespacio $D$ al punto en la proyectivización $\mathbb{P} (\bigwedge^d V)$ correspondiente a $\bigwedge^d D$ ($d \geq 1$). Afirmamos que $\psi$ es inyectiva. En efecto, supongamos que $D,D'$ son dos subespacios $d$-dimensionales. Fijemos una base de $V$ de tal suerte que $v_1, \ldots, v_d$ genera $D$, mientras que $v_{r, \ldots, v_{r + d - 1}}$ genera $D'$. De esta manera $v_1 \wedge \cdots \wedge v_d$ no puede ser un múltiplo de $v_r \wedge \cdots \wedge v_{r+d - 1}$ a menos que $r = 1$, i.e. $D = D'$.

Para poder brindarle a $\mathfrak G _d (V)$ una estructura de variedad proyectiva, basta ver que la imagen de $\psi$ es cerrada. Para esto, basta probarlo en cada cubrimiento afín de $\mathbb{P}(\bigwedge^d (V))$. Los casos $d = 1, d = n$ son triviales.

Fijemos una base ordenada $(v_1, \ldots, v_n)$ de $V$, y asosciemos la base $\{v_{i_1} \wedge \cdots \wedge v_{i_d}\}_{i_1 < \cdots < i_d}$ de $\bigwedge^d V$. Consideremos los abiertos afines $U$ que cubren $\mathbb{P} (\bigwedge^d (V))$ que consisten de puntos cuya coordenada homogénea relativa a $v_{i_1} \wedge \cdots \wedge v_{i_d}$ ($i_1 < \cdots < i_d$) es no nula. Por ejemplo, veamos por simplicidad el caso dado por $v_1 \wedge \cdots \wedge v_d$. Veamos ahora que la intersección $\Imagen \psi \cap U$ es cerrada en $U$. Pongamos $D_0$ como el subespacio generado por $v_1, \ldots, v_d$. Claramente $\psi (D)$ pertenece a $U$ si y solo si la proyección natural de $V$ sobre $D_0$ manda $D$ de manera isomorfa a $D_0$. En este caso, las imágenes inversas de $v_1, \ldots, v_d$ forman una base de $D$ de la forma $v_i + x_i (D)$ donde $x_i (D) := \sum_{j > d} a_{ij} v_j$ (y esta es la única base de $D$ que tiene esta forma). El producto cuña luce de la siguiente forma
\[
    v_1 \wedge \cdots \wedge v_d + \sum_{i = 1}^d (v_1 \wedge \cdots \wedge x_i (D) \wedge \cdots \wedge v_d) + (\star),
\]
donde $(\star)$ involucre una base de vecctores con dos o más $v_1, \ldots, v_d$ omitidos. Aquí se tiene que $v_1 \wedge \cdots \wedge x_i (D) \wedge \cdots \wedge v_d = \sum_{j > d} a_{ij} (v_1 \wedge \cdots \wedge v_j \wedge \cdots \wedge v_d)$ con $v_j$ reemplazado por $v_i$. Por lo tanto $\pm a_{ij}$ ($ 1 \leq i \leq d, d + 1 \leq j \leq n$) puede recuperarse como el coeficiente de la base de elementos $v_1 \wedge \cdots \wedge \hat {v_i} \wedge \cdots \wedge v_d \wedge v_j$ ($v_i$ omitido), en el producto cuña de la base de $D$ dada arriba. Más aún, los coeficientes en $(\star)$ son obviamente funciones polinomiales de los $a_{ij}$, independientes de $D$.

Recíprocamente, dados $d (n-d)$ escalares $a_{ij}$ arbitrarios, claramente los vectores resultantes $v_i + x_i (D)$ generan un subespacio $d$-dimensional de $V$, cuya imagen bajo $\psi$ yace en $U$. Consecuentemente, $\Imagen \psi \cap U$ consiste de todos los puntos cuyas coordenadas afines son $(\ldots a_{ij} \ldots, f_k (a_{ij}) \ldots)$, donde los $a_{ij}$ son arbitrarios y los $f_k$ son funciones polinomiales en $\afine {d (n-d)}$. Este conjunto se puede ver como el \textit{grafo} de un morfismo de $\afine{d (n-d)}$ en otro espacio afín. Como los grafos son cerrados en la \textit{topología Zariski producto} (\cite[Teorema 2.6.12.]{notas_pedro}), concluimos que $\Imagen \psi \cap U$ es cerrado en $U$.

\begin{definition}
Las \textbf{variedades Grassmannianas} son los $\mathfrak G _d (V)$. Una \textbf{bandera} en $V$, es por definición una cadena
\[
    0 \subset V_1 \subset \cdots \subset V_k = V
\]
de subespacios $k$-vectoriales de $V$, tales que las inclusiones son propias. Una \textbf{bandera completa} es una en la que $k = \dim V$, i.e., $\dim V_{i+1}/V_i = 1$. $\mathfrak F (V)$ denota al conjunto de todas las banderas completas de $V$. Una vez que brindemos a $\mathfrak F (V)$ una estructura de variedad proyectiva, llamaremos a estas variedades: \textbf{variedad bandera} de $V$.
\end{definition}


Como vimos en \cite[Corolario 2.7.17.]{notas_pedro}, el producto de grassmanianas $\mathfrak G_1 (V) \times \cdots \times \mathfrak G_n (V)$ tiene la estructura de variedad proyectiva. Por otro lado, $\mathfrak F (V)$ se identifica con un subconjunto en este producto de grassmanianas de manera obvia, luego para darle una estuctura de variedad proyectiva basta ver que este conjunto es cerrado. Para simplificar la notación, consideremos solamente el producto $\mathfrak G _d (V) \times \mathfrak G_{d+1} (V)$, y probemos que el conjunto $S$ de pares $(D, D')$ con $D \subset D'$ es cerrado.

Nuevamente como antes, fijemos una base ordenada $(v_1, \ldots, v_n)$ de $V$, y consideremos dos abiertos afines de $\mathbb{P} (\bigwedge^d (V))$ y $\mathbb{P} (\bigwedge^{d+1} (V))$, de tal suerte que los productos cubren la vareidad $\mathfrak G _d (V) \times \mathfrak G _{d+1} (V)$. Otra vez para simplificar la notación, tomemos estos abiertos afines $U, U'$ definidos relativos a $v_1 \wedge \cdots \wedge v_d$, $v_1 \wedge \cdots \wedge v_{d+1}$ respectivamente. Si $D$ (respectivamente $D'$) es la imagen en $U$ (respectivamente $U'$), obtenemos como ante bases canónicas: $v_i + x_i (D)$ ($1 \leq i \leq d$); $v_i + y_i (D')$ ($1\leq i \leq d+ 1$). Con lo cual $x_i (D) = \sum_{j > d} a_{ij} v_j$, $y_i (D') = \sum_{j > d+1} b_{ij} v_j$. Notemos que $D \subset D'$ si y solo si $x_i (D) = y_i (D') + a_{i, d+1} (v_{d+1} + y_{d+1} (D'))$ para $1 \leq i \leq d$. Esto a su vez se convierte en condiciones polinomiales en los coeficientes $a_{ij}, b_{ij}$, con lo cual $S$ interseca $U \times U'$ en un cerrado de $U \times U'$.





















\section{Introducción a Grupos Algebraicos}

Sea $G$ una variedad algebraica dotada de estructura de grupo. Si los mapas
\[
    \begin{array}{ll}
    \mu : G \times G &\longrightarrow G \\
    (x,y) &\longmapsto xy
    \end{array}, \quad
    \begin{array}{ll}
    \iota : G &\longrightarrow G \\
    x &\longmapsto x^{-1}
    \end{array}
\]
son morfismos de variedades, decimos que $G$ es un \textbf{grupo algebraico}. Notar que un grupo algebraico \underline{no es un grupo topológico}, excepto cuando la dimensión es $0$. En efecto, para grupos topológicos, $T_1$ es quivalente a $T_2$, con lo cual si $G$ fuera un grupo topológico con la topología Zariski, entonces sería de dimensión $0$.

Trasladar por un elemento $y \in G$, i.e. $x \mapsto x y$, es un isomorfismo de variedades algebraicas $G \to G$, entonces todas las propiedades geométricas que sucedan en un punto de $G$ se transfieren a cualquier otro punto al hacer variar $y$. Por ejemplo, como $G$ tiene algún punto suave (el conjunto $G_{\textup{reg}}$ es un abierto denso de $G$ por \cite[Teorema 2.13.12.]{notas_pedro}), todos los puntos de $G$ son suaves, es decir, $G$ es suave.

Definimos un \textbf{isomorfismo} de grupos algebraicos $G$ y $G'$, como un isomorfismo de variedades algebraicas $\varphi : G \to G'$ el cual es a su vez un isomorfismo de grupos. Así, un \textbf{automorfismo} de $G$ es un isomorfismo de $G$ en $G$. A partir de ahora nos concentraremos en el caso en el cual la estructura de variedad de $G$ sea afín, y seguiremos llamandolá ``grupo algebraico''.

\begin{example}
El \textbf{grupo aditivo} $\mathbb{G}_a$ es la recta afín $\afine 1$ con la estructura de grupo dada por la suma $\mu (x,y) = x + y$ y $\iota (x) = -x$. El \textbf{grupo multiplicativo} $\mathbb{G}_m$ es el abierto afín $k^\times \subset \afine 1$ con la estructura de grupo dada por la multiplicación $\mu (x,y) = xy$ y $\iota (x) = x^{-1}$. Notar que ambos grupos son variedades irreducibles de dimensión $1$.

Más en general, $\afine n$ es un grupo algebraico con la estructura aditiva.
\end{example}

\begin{example}
Sea $\operatorname{GL}_n (k)$ el conjunto de todas las matrices invertibles con coeficientes en $k$ y de tamaño $n \times n$. Es un grupo con la multiplicación de matrices, y se lo suele llamar el \textbf{grupo general lineal}. El conjunto $M_n (k)$ de las matrices con coeficientes en $k$ y de tamaño $n \times n$ puede ser identificado con $\afine {n^2}$, y $\operatorname{GL}_n (k)$ con el abierto principal definido por la función polinomial $\det$. Así, visto como una variedad afín, $\operatorname{GL}_n (k)$ tiene su álgebra de funciones regulares generada por las restricciónes de las $n^2$ funciones coordenada $T_{ij}$ junto con $1 / \det (T_{ij})$. Claramente la multiplicación e inversión de matrices son morfismos de variedades algebraicas, y por lo tanto $\operatorname{GL}_n (k)$ es un grupo algebraico.
\end{example}

\begin{obs}
Un subgrupo cerrado de un grupo algebraico es nuevamente un grupo algebraico.
\end{obs}
\begin{proof}
Sea $H \leq G$ un subgrupo cerrado para la topología Zariski de un grupo algebraico $G$, entonces $H$ tiene una estructura de variedad inducida por $G$. Más aún, los morfismos $\mu : G \times G \to G$ y $\iota : G \to G$ se restringen y corestringen a $H \times H \to H$ y $H \to H$, respectivamente, y siguen siendo morfismos regulares.
\end{proof}

Gracias a esta observación, podemos construir más ejemplos de grupos algebraicos:
\begin{example}
El grupo de \textbf{matrices triangulares superiores} (análogamente las triangulares inferiores) $T_n (k)$ es el conjunto de ceros en $\operatorname{GL}_n (k)$ de los polinomios $T_{ij}$ con $i >j$. También tenemos los subgrupos $D_n (k)$ y $U_n (k)$ de las matrices diagonales y las matrices triangulares superiores con entradas $1$ en la diagonal son cerrados de $\operatorname{GL}_n (k)$. En efecto, como recién, las matrices diagonales es el conjunto de ceros de $T_{ij}$ con $i \neq j$ y las matrices de $U_n (k)$ son los ceros de $T_{ij}$ con $i > j$ y $T_{ii} - 1$.
\end{example}

\begin{obs}
El \textbf{producto directo} de finitos grupos algebraicos es nuevamente un grupo algebraico con la topología producto Zariski y la estructura de grupo dada por el producto cartesiano de grupos.
\end{obs}
\begin{proof}
Veamos solo el caso de producto de dos grupos algebraicos $G_1, G_2$, el caso general se sigue por inducción. Solo tenemos que probar que
\[
    \begin{array}{ll}
    \mu : G_1 \times G_2 \times G_1 \times G_2 &\longrightarrow G_1 \times G_2 \\
    ((x_1,x_2),(y_1,y_2)) &\longmapsto (x_1 y_1, x_2 y_2)
    \end{array}, \quad
    \begin{array}{ll}
    \iota : G_1 \times G_2 &\longrightarrow G_1 \times G_2 \\
    (x_1, x_2) &\longmapsto (x_1^{-1}, x_2^{-1})
    \end{array}
\]
son morfismos regulares. Pero por la propiedad universal del producto de variedades (ver \cite[Teorema 2.6.5.]{notas_pedro}) basta ver que estos mapas son regulares cuando componemos por la proyección a la primera o segunda coordenada. Esto es obvio, pues $G_1$ y $G_2$ son grupos algebraicos, con lo cual sus respectivos mapas $\mu$ y $\iota$ son regulares.
\end{proof}

Finalmente, mencionamos algunos otros nombres de grupos algebraicos clásicos: el \textbf{grupo especial lineal} $\operatorname{SL}_n (k)$ de matrices de $\operatorname{GL}_n (k)$ con determinante $1$ (es el conjunto de ceros de $\det (T_{ij}) - 1$); el \textbf{grupo simpléctico} $\operatorname{Sp}_{2n} (k)$ dada por las matrices $x \in \operatorname{GL}_{2n} (k)$ tales que ${}^t x \begin{pmatrix} 0 & J \\ - J & 0 \end{pmatrix} x = \begin{pmatrix} 0 & J \\ - J & 0 \end{pmatrix}$, donde $J$ es la matriz de tamaño $n \times n$ que tiene $1$ en la antidiagonal y ceros en el resto de los lugares, y ${}^t x$ es la matriz transpuesta de $x$ (es el conjunto de ceros dada por ciertas condiciones polinomiales en $x$); el \textbf{grupo especial ortogonal} $\operatorname{SO}_{2n + 1} (k)$ que en $\Char k \neq 2$ consiste de las matrices $x \in \operatorname{SL}_{2n+1} (k)$ tales que ${}^t x s x = s$, donde $s = \begin{pmatrix} 1 & 0 & 0 \\ 0 & 0 & J \\ 0 & J & 0 \end{pmatrix}$, y también exist otro \textbf{grupo especial ortogonal} $\operatorname{SO}_{2n} (k)$ dada por ${}^t x s x = s$, donde ahora $s = \begin{pmatrix} 0 & J \\ J & 0 \end{pmatrix}$ si $\Char k \neq 2$\footnote{En general los grupos simpléctico y especial ortogonales surgen geométricamente de grupos de transformaciones lineales que preservan ciertas formas bilineales, pero en caractéristica $\Char k = 2$ hay que tener cierto cuidado al definirlos, cf \cite{dieudonne1956groupesDeLieEtHyperalgebresDeLieSurUnCorpsDeCaracteristiqueP}, \cite[Ch. 1]{carter1989simpleGroupsOfLieType}.}.




\section{La componente de la identidad}

Sea $G$ un grupo algebraico. Afirmamos que una sola componente irreducible de $G$ (como variedad algebraica) puede pasar por $1$. En efecto, sea $X_1, \ldots, X_m$ las componentes distintas que contengan a $1$. La imagen de la variedad irreducible $X_1 \times \cdots \times X_m$ en $G$ bajo el morfismo producto es un subconjunto irreducible $X_1 \cdots X_m$ de $G$, que vuelve a contener a $e$. Por lo tanto $X_1 \cdots X_m$ está contenido en algún $X_i$. Por otro lado, claramente todas las componentes $X_1, \ldots, X_m$ están contenidas en $X_1 \cdots X_m$. Esto fuerza a que $m = 1$. Con lo cual, la siguiente definición tiene sentido:

\begin{definition}
Sea $G$ un grupo algebraico. Denotamos por $G^\circ$ a la única componente irreducible de $G$ que contiene a $1$. La llamaremos \textbf{componente de la identidad} de $G$.
\end{definition}

\begin{proposition}
Sea $G$ un grupo algebraico. Entonces:
\begin{enumerate}[(1)]
\item $G^\circ$ es un subgrupo normal cerrado de índice finito en $G$, cuyos cosets son las componentes conexas, como también las componentes irreducibles de $G$.
\item Todo subgrupo cerrado de índice finito de $G$ contiene a $G^\circ$.
\end{enumerate}
\end{proposition}
\begin{proof}
\begin{enumerate}[(1)]
\item Para cada $x \in G^\circ$, $x^{-1} G^\circ$ es una componente irreducible de $G$ que contiene a $1$, luego por unicidad $x^{-1} G^\circ = G^{\circ}$. Es decir, $G^\circ = (G^{\circ})^{-1}$ y por lo tanto $G^\circ G^\circ = G ^\circ$, i.e., $G^\circ$ es un subgrupo cerrado (es una componente irreducible) de $G$. Para todo $x \in G$, nuevamente $x G^{\circ} x^{-1}$ es una componente irreducible de $G$ que contiene a $1$, con lo cual la unicidad fuerza a que $x G^\circ x^{-1} = G^{\circ}$, i.e., $G^\circ$ es normal en $G$. Sus cosets a izquierda o derecha son traslaciones de $G^\circ$, y por lo tanto deben ser componentes irreducibles de $G$; solamente puede haber un número finito de estos ($G$ es un espacio Noetheriano; ver \cite[Teorema 2.8.9.]{notas_pedro}), en particular $G^\circ$ tiene índice finito. Como estos conjuntos son disjuntos, deben ser también componentes conexas de $G$, ya que los conjuntos irreducibles en un espacio topológico son conexos, y al ser finitos cerrados (son componentes irreducibles) disjuntos también son abiertos.
\item Si $H$ es un subgrupo cerrado de índice finito en $G$, entonces cada uno de sis finitos cosets a izquierda son también cerrados, con lo cual la unión de los cosets distintos de $H$ también lo es. Como el complemento de un cerrado es abierto, tenemos que $H$ es abierto. Consecuentemente, los cosets a izquierda de $H$ particionan $G^\circ$ en una unión finita de abiertos-cerrados, como $G^\circ$ es conexo e interseca $H$, debe ser que $G^\circ \subset H$.
\end{enumerate}
\end{proof}

Esto motiva a la siguiente definición:
\begin{definition}
Diremos que un grupo algebraico es \textbf{conexo}\footnote{El uso del término ``irreducible'' tiene reservado un significado totalmente distinto en el contexto de grupos lineales y representaciones de grupos.} si $G = G^\circ$.
\end{definition}

\begin{example}
La mayoría de los grupos algebraicos que hemos mencionado son conexos: $\mathbb G_a$, $\mathbb{G}_m$, $\operatorname{GL}_n (k), \operatorname{SL}_n (k)$. Que $\operatorname{GL}_n (k)$ es conexo es consecuencia de ser un abierto principal del espacio afín $\afine {n^2}$. Sin embargo, probar tanto que $\operatorname{SL}_n (k)$ como otros grupos clásicos son conexos no se deduce trivialmente de la definición de estos grupos (ver \cite[(7.5)]{humphreys2012linearAlgebraicGroups}).
\end{example}














\section{Resumen de variedades completas}

\begin{definition}
Decimos que una variedad algebraica (o simplemente variedad) $X$ es \textbf{completa}, si para toda variedad algebraica $Y$, la proyección a la segunda coordenada
\begin{align*}
\operatorname{pr}_Y : X \times Y &\longrightarrow Y \\
(x,y) &\longmapsto y
\end{align*}
es una función \textit{cerrada}.
\end{definition}

\begin{proposition}\label{proposicion:varios hechos sobre variedades completas}
\begin{enumerate}[(a)]
\item Una subvariedad cerrada de una variedad completa (respectivamente proyectiva) es completa (respectivamente proyectiva).
\item Si $\varphi : X \to Y$ es un morfismo (regular) de variedades algebraicas, y $X$ es completo, entonces la imagen es cerrada en $Y$, y es completa.
\item Las variedades afines completas tienen dimensión $0$.
\item Las variedades proyectivas son completas; las variedades quasiproyectivas completas son proyectivas.
\item La \textit{variedad bandera} de un espacio vectorial $V$ de dimensión finita es proyectiva, y en particular el ítem (d) dice que es completa.
\end{enumerate}
\end{proposition}
\begin{proof}
\begin{enumerate}[(a)]
\item Se deduce inmediatamente de la definición de subvariedad cerrada.
\item Es exactamente la misma demostración que el Corolario 2.7.10. de \cite{notas_pedro}; notar que usamos que la variedad algebraica $Y$ es \textit{separada}.
\item En efecto, como $X$ es afín, podemos suponer sin pérdida de generalidad que es un cerrado de $\afine m$ para algún $m \geq 1$, luego basta ver que la imagen de cada proyección a la $i$-ésima coordenada es finita, digamos $f_i : X \to \afine 1$, ahora, considerando la incrustación $\afine 1 \hookrightarrow \projective 1$, $x \mapsto [x, 1]$, tenemos que por el ítem anterior que la composición $X \to \afine 1 \hookrightarrow \projective 1$ tiene imagen cerrada, y como no es sobreyectiva, debe ser finita, i.e., la imagen de $f_i$ es finita como queríamos probar.
\item Que las \textit{variedades proyectivas} son completas ya lo vimos en \cite[Teorema \emph{2.7.9}]{notas_pedro}. Más generalmente, si $X$ es quasi-proyectiva, es decir, isomorfa a un abierto Zariski $U$ de una variedad algebraica proyectiva $Y$, entonces el morfismo inclusión $U \hookrightarrow Y$ tiene imagen cerrada por el ítem (b), y por lo tanto es una subvariedad cerrada de una variedad proyectiva, y concluimos utilizando el ítem (a).
\item Una demostración de que las variedades banderas son proyectivas se puede encontrar en \cite[Teorema 3.3.11.]{geckMeinolf2013introductionToAlgebraicGeometryAndAlgebraicGroups}.
\end{enumerate}
\end{proof}

Además de estos hechos, necesitamos un lema:

\begin{lemma}\label{lema:lema para variedades completas}
Supongamos que $G$ actua transitivamente sobre dos variedades algebraicas irreducibles $X$,$Y$, y sea $\varphi : X \to Y$ un morfismo regular biyectivo, $G$-equivariante. Si $Y$ es completo, entonces $X$ también.
\end{lemma}
\begin{proof}
Ver \cite[Lema \S 21.1.]{humphreys2012linearAlgebraicGroups}.
\end{proof}


\section{Teorema del punto fijo de Borel}

En esta sección probaremos el teorema principal de este artículo:

\begin{theorem}[Teorema del punto fijo de Borel]\label{teorema:punto fijo de borel}
Sea $G$ un grupo algebraico conexo soluble, y sea $X$ una variedad completa (no vacía) donde actúa $G$. Entonces $G$ tiene un punto fijo en $X$.
\end{theorem}
\begin{proof}
Si $\dim G = 0$, entonces $G = \{1\}$ y no hay nada que probar. Luego procedemos por inducción en la dimensión de $G$. Sea $H := [G, G]$, es conexo (ver \cite[(17.2)]{humphreys2012linearAlgebraicGroups}), soluble, y de menor dimensión que $G$ (pues $G$ es soluble), con lo cual, por hipótesis inductiva, el conjunto $Y$ de puntos fijos de $H$ en $X$ es no vacío. $Y$ es cerrado (ver \cite[Proposición 8.2.]{humphreys2012linearAlgebraicGroups}), con lo cual es completo por el ítem (a) de la Proposición \ref{proposicion:varios hechos sobre variedades completas}. Como $G$ deja estable a $Y$, ya que $H$ es normal en $G$, basta ver que $G$ tiene un punto fijo en $Y$, así, reemplacemos $X$ por $Y$.

Estamos entonces en el siguiente caso: $H \subset G_x$ para todo $x \in X$. En particular, todos los \textit{grupos de isotropía} son normales en $G$, por lo tanto $G/G_x$ es una variedad \textit{afin}. Que los grupos de isotropía son normales se deducen de lo siguiente, esto equivale a probar que para todo $g \in G$, $G_x \subset g G_x g^{-1}$, luego sea $z \in G_x$, i.e., $z \cdot x = x$, tenemos que $g^{-1}z g z^{-1} \in H \subset G_{z \cdot x}$, consecuentemente $x = z \cdot x = g^{-1} z g z^{-1} (z \cdot x) = g^{-1} z g \cdot x$, i.e. $g ^{-1} z g \in G_x$, o sea, $z \in g G_x g^{-1}$, como $z$ era arbitrario se prueba la inclusión deseada.

Tomemos $x \in X$ cuya órbtia $G \cdot x$ sea cerrada, y por lo tanto nuevamente completo: esto se puede hacer, por \cite[Proposición 8.3.]{humphreys2012linearAlgebraicGroups}. Ahora el morfismo canónico $G / G_x \to G \cdot X$ es biyectivo, con el lado izquierdo afín y el derecho completo. El Lema \ref{lema:lema para variedades completas} implica que $G/G_x$ es completo. Pero el ítem (c) de \ref{proposicion:varios hechos sobre variedades completas} implica que $G/G_x$ es un grupo algebraico $0$-dimensional, i.e. trivial, es decir, $G_x = G$, y por lo tanto $x$ es un punto fijo.
\end{proof}

\section{Consecuencias}

En esta sección probaremos varias consecuencias el Teorema del punto fijo de Borel. Sea $G$ un grupo conexo arbitrario.

El siguiente teorema es un análogo del Teorema de Lie\footnote{Este teorema dice que sobre un cuerpo algebraicamente cerrado de \underline{característica cero}, si $\mathfrak g \to \mathfrak {gl} (V)$ es una representación de dimensión finita de una álgebra de Lie \textit{soluble} $\mathfrak g$, entonces existe una \textit{bandera} $V = V_0 \supset V_1 \supset \cdots \supset V_n = 0$ de subespacios $\mathfrak g$-invariantes con $\operatorname{codim} V_i = i$. En particular, $V_{n-1}$ es $1$-dimensional, y por lo tanto contiene un vector $v$ que es autovector simultáneo de cada $\pi (g)$ para todo $g \in \mathfrak g$.}; esto vale en característica arbitraria, sin embargo el teorema para álgebras de Lie \underline{no}\footnote{En caaracterística $p > 0$, el Teorema de Lie vale para representaciones de dimensión menor que $p$, sin embargo, puede fallar en dimensión $p$: ver \cite{wikipedia-lie-theorem}.}.
\begin{theorem}[Teorema de Lie-Kolchin]\label{teorema:Lie-Kolchin}
Sea $G$ un subgrupo algebraico conexo soluble de $\operatorname{GL} (V)$ para un espacio vectorial de dimensión finita no trivial. Entonces existe una bandera $V = V_0 \supset V_1 \supset \cdots \supset V_n = 0$ de subespacios $G$-invariantes con $\operatorname{codim} V_i = i$. En particular, $V_{n-1}$ es $1$-dimensional, y por lo tanto contiene un vector $v$ que es autovector simultáneo de cada $g$ para todo $g \in G$.
\end{theorem}
\begin{proof}
Sea $G$ un subgrupo cerrado conexo soluble de $\operatorname{GL} (V)$. Entonces $G$ actúa en la variedad bandera de $V$, la cual es completa por el ítem (e) de la Proposición \ref{proposicion:varios hechos sobre variedades completas}, con lo cual el Teorema \ref{teorema:punto fijo de borel} implica que la acción de $G$ deja fija una bandera
\[
    V = V_n \supset \cdots V_1 \supset V_0 = 0.
\]
Es decir, vale el enunciado del teorema.
\end{proof}

\subsubsection{Subgrupos de Borel y Toros maximales}

\begin{definition}
Un \textbf{subgrupo de Borel} de $G$ es un subgrupo cerrado conexo soluble que no está incluido propiamente en ningún otro subgrupo cerrado conexo soluble.
\end{definition}
Como los subgrupos de Borel de $G$ y $G^{\circ}$ coinciden, supondremos a partir de lo que sigue que $G$ es conexo. Un subgrupo conexo soluble de dimensión máxima en $G$ es claramente un subgrupo de Borel; pero no es obvio que todo subgrupo de borel tenga la misma dimensión, sin embargo, esto es cierto:

\begin{theorem}\label{teorema:los subgrupos de borel son conjugados y G/B es una variedad proyectiva}
Sea $B$ un subgrupo de borel de $G$. Entonces $G/B$ es una variedad proyectiva, y todos los otros subgrupos de borel son conjugados a $B$. En particular, son todos isomorfos entre sí y tienen la misma dimensión.
\end{theorem}
\begin{proof}
Sea $S$ un subgrupo de Borel de dimensión máxima. Representemos a $G$ en $\operatorname{GL} (V)$ con un subespacio $1$-dimensional $V_1$ cuyo estabilizador en $G$ es precisamente $S$ (ver \cite[Teorema 11.2]{humphreys2012linearAlgebraicGroups}). La acción inducida de $S$ en $V/V_1$ es \textit{trigonalizable} por el Teorema \ref{teorema:Lie-Kolchin}, co lo cual existe una bandera \textit{completa} $0 \subset V_1 \subset \cdots \subset V$ estabilizada por $S$, llamemoslá $f$. De hehco, $S$ es el grupo de isotropía de $f$ en $G$, por cómo elegimos $V_1$. Con lo cual el morfismo inducido de $G/S$ sobre la órbita de $f$ en la variedad bandera de $V$ es biyectiva. Por otro lado, el estabilizador de toda variedad bandera es soluble y por lo tanto tiene dimensión no mayor a $\dim S$. Consecuentemente, la órbita de $f$ tiene la dimensión más chica posible, por lo tanto es cerrado (ver \cite[(8.3)]{humphreys2012linearAlgebraicGroups}). Así, la órbita es completa por los ítems (a) y (e) de la Proposición \ref{proposicion:varios hechos sobre variedades completas}. Esto fuerza a que $G/S$ sea completo por el Lema \ref{lema:lema para variedades completas}, o sea, es proyectivo por el ítem (d) de \ref{proposicion:varios hechos sobre variedades completas}.

Ahora tomemos un subgrupo de Borel $B$, este actúa por multiplicación a izquierda en la variedad completa $G/S$. Luego el Teorema \ref{teorema:punto fijo de borel} implica que deja fijo un punto $xS$, es decir, $BxS = xS$, equivalentemente, $x^{-1} B x \subset S$. Como ambos son subgrupos de Borel, concluimos que $x^{-1} B x = S$ por maximalidad. Esto concluye ambas afirmaciones del teorema.
\end{proof}

\begin{corollary}\label{corolario:los toros maximales de G son toros maximales de un Borel y son todos conjugados entre si}
Los toros maximales (respectivamente los subgrupos conexos unipotentes maximales) de $G$ son los de los subgrupos de Borel de $G$, y son todos conjugados. En particular tienen la misma dimensión.
\end{corollary}
\begin{proof}
Sea $T$ un toro maximal de $G$, $U$ un subgrupo conexo unipotente maximal. Evidentemente $T$ está inclluido en algún subgrupo de borel $B$, entonces es un toro maximal de $B$, y por lo tanto todos los demás toros maximales de $B$ son conjugados a $T$ en $B$ (ver \cite[Teorema 19.3]{humphreys2012linearAlgebraicGroups}). Similarmente, $U$ yace contenido en algún subgrupo de borel $B'$, con $U = B_u '$ por maximalidad. Como todos los subgrupos de Borel de $G$ son conjugados, el corolario se sigue.
\end{proof}

\begin{definition}
A la dimensión de cualquier toro maximal de $G$ se le dice el \textbf{rango} de $G$.
\end{definition}

\begin{example}
El rango de $\operatorname{SL}_n (k)$ es $n-1$.
\end{example}

\begin{definition}
Decimos que un subgrupo cerrado $P$ de $G$ es \textbf{parabólico}, si el \textit{espacio homogéneo} $G/P$ es proyectivo (equivalentemente completo por el ítem (d) de \ref{proposicion:varios hechos sobre variedades completas}).
\end{definition}

\begin{corollary}\label{corolario:un subgrupo cerrado es parabolico si y solo si incluye un subgrupo de borel}
Un subgrupo cerrado $P$ de $G$ es parabólico si y solo si contiene un subgrupo de Borel. En particular, todo subgrupo conexo $H$ de $G$ es un subgrupo de Borel si y solo si $H$ es soluble y $G/H$ es proyectivo.
\end{corollary}
\begin{proof}
Si $H$ es un subgrupo cerrado de $G$ tal que $G/H$ es proyectivo, entonces $B$ deja fijo un punto por el Teorema \ref{teorema:punto fijo de borel}, y por lo tanto tiene un conjugado en $H$, esto fuerza a que $\dim G/H \leq \dim G/B$. Recíprocamente, si $H$ es un subgrupo cerrado incluyendo un subgrupo de Borel $B$ de $G$, entonces $G/B \to G/H$ es un morfismo sobreyectivo con dominio una variedad comleta, forzando a $G/H$ a ser completo (ítem (b) de la Proposición \ref{proposicion:varios hechos sobre variedades completas}). Pero $G/H$ es proyectivo (ítem (d) de la Proposición \ref{proposicion:varios hechos sobre variedades completas}), ya que todos los espacios homogéneos son quasi-proyectivos por construcción (ver \cite[(11.3)]{humphreys2012linearAlgebraicGroups}). Esto prueba el Corolario.
\end{proof}

\begin{example}
Sea $G = \operatorname{GL}_n (k)$ y $B = \operatorname{T}_n (k)$ las matrices triangulares superioes de $G$. El Teorema de Lie-Kolchin \ref{teorema:Lie-Kolchin} dice que $B$ es un subgrupo de Borel de $G$. En efecto, $G/B$ es justamente la órbita en la variedad bandera de $V = k^n$ de la \textit{bandera standard}. Calculemos los subgrupos parabólicos de $G$ que contienen a $B$. Si $(e_1, \ldots, e_n)$ es la base canónica de $k^n$, entonces para cada bandera parcial $(e_1, \ldots, e_{i(1)}) \subset (e_1, \ldots, e_{i(2)}) \subset \cdots$, el estabilizador de $G$ es evidentemente un subgrupo cerrado incluyendo a $B$.

Más concretamente, si $G = \operatorname{GL}_3 (k)$, entonces existen solamente dos subgrupos parabólicos propios distintos de $B$: los dos grupos de matrices de la siguiente forma
\[
    \begin{pmatrix}
    * &* & * \\
    * & * & * \\
    0 & 0 & *
    \end{pmatrix}, \quad
    \begin{pmatrix}
    * &* & * \\
    0 & * & * \\
    0 & * & *
    \end{pmatrix}.
\]
\end{example}

\begin{corollary}\label{corolario:los epimorfismos preservan subgrupos}
Sea $\varphi : G \to G'$ un epimorfismo de grupos algebraicos conexos. Sea $H$ un subgrupo de Borel (respectivamente un subgrupo parabolico, un toro maximal, o un subgrupo conexo unipotente maximal) de $G$. Entonces $\varphi (H)$ es un subgrupo  de Borel (respectivamente un subgrupo parabolico, un toro maximal, o un subgrupo conexo unipotente maximal) de $G'$ y todos los subgrupos de este tipo en $G'$ se obtienen de esta manera.
\end{corollary}
\begin{proof}
Debido a los Corolarios \ref{corolario:los toros maximales de G son toros maximales de un Borel y son todos conjugados entre si}, \ref{corolario:un subgrupo cerrado es parabolico si y solo si incluye un subgrupo de borel}, basta ver que esto vale en el caso $H = B$ subgrupo de borel de $G$. Claramente $B' := \varphi (B)$ es conexo y soluble. Pero el mapa natural $G \to G' \to G'/B'$ induce un morfismo sobreyectivo $G/B \to G'/B'$, y por lo tanto $G'/B'$ es completo por el ítem (b) de la Proposición \ref{proposicion:varios hechos sobre variedades completas}, es decir, $B'$ es un subgrupo parabólico de $G'$. El Corolario \ref{corolario:un subgrupo cerrado es parabolico si y solo si incluye un subgrupo de borel} implica luego que es un subgrupo de Borel de $G'$. Como algún subgrupo de Borel de $G'$ tiene que ser de la forma $\varphi (B)$, se sigue del Teorema \ref{teorema:los subgrupos de borel son conjugados y G/B es una variedad proyectiva} aplicado a $G'$ que son todos conjugados a $\varphi (B)$, consecuentemente, por sobreyectividad de $\varphi$, deben ser de esta forma.
\end{proof}

\subsubsection{Más consecuencias}
En esta subsubsección supongamos que $G$ es un grupo algebraico conexo.


\begin{proposition}
Si $\sigma$ es un automorfismo de $G$ que deja fijo todos los elementos de un subgrupo de Borel $B$, entonces debe ser la identidad.
\end{proposition}
\begin{proof}
El morfismo
\begin{align*}
\varphi : G &\longrightarrow G \\
x &\longmapsto \sigma(x)x^{-1},
\end{align*}
envía $B$ en $1$, y por lo tanto se factoriza por la proyección a través del cociente $G \to G/B$. Por el Teorema \ref{teorema:los subgrupos de borel son conjugados y G/B es una variedad proyectiva} y el ítem (b) de la Proposición \ref{proposicion:varios hechos sobre variedades completas}, $\varphi (G)$ es cerrado (y por lo tanto afín) y completo. Así, es un grupo algebraico $0$-dimensiónal, i.e. $\varphi (G) = \{1\}$.
\end{proof}

\begin{corollary}
\[
    Z(G)^{\circ} \subset Z(B) \subset C_G (B) = Z(G).
\]
\end{corollary}
\begin{proof}
$Z(G)^\circ$ es conexo y soluble, por lo tanto está contenido en algún subgrupo de Borel de $G$, el cual podemos conjugar (sin afectar a $Z(G)^\circ$) para convertirlo en $B$ gracias al Teorema \ref{teorema:los subgrupos de borel son conjugados y G/B es una variedad proyectiva}, con lo cual la primera inclusión vale. La segunda inclusión es obvia; también lo es $Z(G) \subset C_G (B)$. Finalmente, si $x \in C_G (B)$, entonces el autormofismo $\sigma : g \mapsto x g x^{-1}$ cumple las hipótesis de la proposición anterior, con lo cual es trivial, i.e., $x \in Z(G)$.
\end{proof}

\begin{remark}
De hecho, se puede probar que $Z(G) = Z(B)$ (ver \cite[Corolario B \S 22.2.]{humphreys2012linearAlgebraicGroups}).
\end{remark}

\begin{proposition}
\begin{enumerate}[(1)]
\item Si un subgrupo de borel $B$ es nilpotente (en particular, si $B = B_s$ o $B = B_u$), entonces $G = B$.
\item $G$ es nilpotente si y solo si $G$ tiene un único toro maximal.
\end{enumerate}
\end{proposition}
\begin{proof}
\begin{enumerate}[(1)]
\item Observemos primero que si $B = B_s$ o $B = B_u$, entonces $B$ es un toro o un grupo unipotente (ver \cite[Teorema 19.3]{humphreys2012linearAlgebraicGroups}), y por lo tanto nilpotente en cualquier ccacso. Siempre que $B$ sea nilpotente, de dimensión positiva, su centro también tiene dimensión positiva (\cite[Proposición 17.4 (a)]{humphreys2012linearAlgebraicGroups}). Sin embargo, el corolario anterior dice que $Z(G)^\circ \subset Z (B) \subset Z (G)$. Con lo cual, podemos pasar a un grupo de menor dimensión $G/Z(G)^{\circ}$, en donde $B / Z(G)^\circ$ es un subgrupo nilpotente de Borel. Por hipótesis inductiva, estos grupos son iguales, con lo cual $G = B$ (el caso base $\dim G = 0$ es trivial).
\item Sabemos por \cite[(19.2)]{humphreys2012linearAlgebraicGroups} que un grupo nilpotente tiene un único toro maximal. Recíprocamente, si $T$ es el único toro maximal de $G$ y $B$ es algún subgrupo de borel conteniéndolo, debe ser que $B$ es nilpotente \cite[(19.2)(19.3)]{humphreys2012linearAlgebraicGroups}, con lo cual $B = G$ por el ítem (1).
\end{enumerate}
\end{proof}

\begin{corollary}
Sea $T$ un toro maximal de $G$, $C := C_G (T)^\circ$. Entonces $C$ es nilpotente y $C = N_G (C)^\circ$.
\end{corollary}
\begin{proof}
$T$ es el único toro maximal de $C$, gracias al Corolario \ref{corolario:los toros maximales de G son toros maximales de un Borel y son todos conjugados entre si}, entonces $C$ es nilpotente por la proposición que acabamos de probar. Claramente $T$ es normal en $N_G (C)^\circ$, y por lo tanto también central (ver \cite[Corolario 16.3]{humphreys2012linearAlgebraicGroups}).
\end{proof}

\begin{definition}
Sea $T$ un toro maximal de $G$, y $C := C_G (T)^\circ$. El grupo $C$ se lo suele llamar \textbf{subgrupo de Cartan}\footnote{En analogía con las subálgebras de Cartan en Teoría de Álgebras de Lie.}.
\end{definition}

\begin{remark}
De hecho, los subgrupos de Cartan son conexos (ver el Teorema \cite[\S 22.3.]{humphreys2012linearAlgebraicGroups}). En particular, como todos los toros maximales son conjugados entre sí, los subgrupos de Cartán también lo son.
\end{remark}


\subsubsection{El Teorema del Normalizador}
Nuevamente $G$ es un grupo algebraico conexo.

Primero probamos un lema:

\begin{lemma}
Sea $B$ un subgrupo de Borel de $G$, sea $N:= N_G(B)$. Entonces $B = N^\circ$.
\end{lemma}
\begin{proof}
Claramente $B$ es un subgrupo de Borel de $N^\circ$. Gracias al Teorema \ref{teorema:los subgrupos de borel son conjugados y G/B es una variedad proyectiva} y al hecho que $B$ es normal en $N^\circ$, $B$ es el único subgrupo de Borel de $N^\circ$. Finalmente, \cite[Teorema de Densidad \S 22.2.]{humphreys2012linearAlgebraicGroups} fuerza a que $B$ sea igual a $N^\circ$.
\end{proof}

\begin{theorem}[Teorema del Normalizador]\label{teorema:teorema del normalizador}
Sea $B$ un subgrupo de Borel de $G$. Entonces $B = N_G (B)$.
\end{theorem}
\begin{proof}
Escribamos $N := N_G (B)$. Por el lema anteior tenemos que $B = N^\circ$. Para probar que $N = B$, procederemos por inducción en $\dim G$. Claramente $R(G)$ yace dentro de todos los subgrupos de Borel de $G$ (cf. \cite[Ejercicio 21.6]{humphreys2012linearAlgebraicGroups}), así que supongamos que sin pérdida de generalidad que $G$ es de dimensión positiva y \textit{semisimple}: si no, aplicamos hipótesis inductiva a $G/R(G)$.

Sea $x \in N$, y sea $T$ un toro maximal de $G$ contenido en $B$. Entonces $x T x^{-1}$ es otro toro maximal de $G$ contenido en $B$, así que existe $y \in B$ tal que $y( x T x^{-1})y^{-1} = T$ (Corolario \ref{corolario:los toros maximales de G son toros maximales de un Borel y son todos conjugados entre si}). Claramente $x $ pertenece a $B$ si y solo si $yx$ lo es, por lo que podemos asumir también que $x \in N_G (T)$. Tomemos $S := C_T (x)^\circ$, un subtoro $T$. Hay dos posibilidades:
\begin{enumerate}
\item[Caso 1:] $S \neq \{1\}$. Luego $C := C_G (S)$ tiene radical no trivial, con lo cual $C$ es un subgrupo propio de $G$. Se tiene que $B' := B \cap C$ es un subgrupo de Borel de $C$ (ver \cite[\S 22.4.]{humphreys2012linearAlgebraicGroups}). Más aún, $C$ es conexo (ver \cite[Teorema 22.3.]{humphreys2012linearAlgebraicGroups}). Por hipótesis inducitva, se sigue que $N_C (B') = B'$. Pero $x$ pertenece a $C$ y normaliza $B$, así que $x \in N_C (B') = B' \subset B$.
\item[Caso 2:] $S = \{1\}$. Como $x$ normaliza $T$, y $T$ es conmutativo, es fácil de ver que el morfismo conmutador
\begin{align*}
\gamma_x : T &\longrightarrow T \\
t &\longmapsto t x t^{-1}x^{-1}
\end{align*}
es de hecho un morfismo de grupos con núcleo $C_T (x)$. Este es finito, ya que $S = \{1\}$, así que $\gamma_x$ debe ser sobreyectivo por un argumento de dimensión (ver \cite[Proposición 7.4B]{humphreys2012linearAlgebraicGroups}). Consecuentemente, $T$ yace dentro de $[M, M]$, donde $M$ es el subgrupo de $N$ generado por $x$ y $B$. Así, $B = T B_u$ yace contenido en el subgrupo de $M$ generado por $B_u$ y $[M,M]$.

Ahora, tomemos una representación racional $\rho : G \rightarrow \operatorname{GL}(V)$, donde $V$ contiene una recta $D$ cuyo grupo de isotropía en $G$ es precisamente $M$ (ver \cite[Teorema 11.2]{humphreys2012linearAlgebraicGroups}). Sea $\chi : M \to \mathbb{G}_m$ su caracter asosciado. Entonces $\chi$ es trivial en $[M,M]$ y en grupo unipotente $B_u$; ergo, $\chi$ es trivial en $B$. Si $0 \neq v \in V$ genera $D$, se tiene que $\rho$ induce un morfismo $G/B \to Y =$ la órbita de $v$ bajo la acción de $\rho (G)$. Como $G/B$ es completo, su imagen $Y$ es cerrada en $V$ (y por lo tanto afín) y también completa gracias al ítem (b) de la Proposición \ref{proposicion:varios hechos sobre variedades completas}. Pero esto implica que $Y$ es un punto por el ítem (c) de la Proposición \ref{proposicion:varios hechos sobre variedades completas}. Entonces $G = M$. Como $G$ es conexo y $[M : B] < \infty$ (lema anterior), no queda otra alternativa que $G$ sea igual a $B$, lo cual es imposible.
\end{enumerate}
\end{proof}

\begin{corollary}
$B$ es maximal en la colección de subgrupos solubles (no necesariamente conexos o cerrados) de $G$.
\end{corollary}
\begin{proof}
Si $S$ es un subgrupo soluble maximal de $G$, entonces evidentemente $S$ es cerrado. Si $S \supset B$, entonces $S^\circ = B$ por definición de subgrupo de Borel, consecuentemente $S \subset N_G (B) = B$.
\end{proof}

\begin{corollary}
Sea $P$ un subgrupo parabólico de $G$. Entonces $P = N_G (P)$. En particular, $P$ es conexo.
\end{corollary}
\begin{proof}
Por definición, $P$ contiene algún subgrupo de Borel $B$ de $G$. Sea $x \in N_G (P)$. Entonces tanto $B$ como $x B x^{-1}$ son subgrupos de Borel de $P^\circ$, y por lo tanto son conjugados por algún elemento $y \in P^\circ$ (Teorema \ref{teorema:los subgrupos de borel son conjugados y G/B es una variedad proyectiva}). Por lo tanto $xy \in N_G (B) = B$ (gracias al Teorema del Normalizador \ref{teorema:teorema del normalizador}). Sin embargo, $xy, y \in P^\circ$ fuerza a que $x \in P^\circ$, es decir, $P^\circ = P = N_G (P)$.
\end{proof}

\begin{corollary}
Sean $P,Q$ subgrupos parabólicos de $G$, ambos conteniendo un subgrupo de Borel $B$. Si $P$ y $Q$ son conjugados en $G$, entonces $P = Q$.
\end{corollary}
\begin{proof}
Supongamos que $Q = x^{-1} P x$. Entonces $B$ y $x^{-1}B x$ son ambos subgrupos de borel del grupo conexo $Q$ (corolario anterior), así existe $y \in Q$ tal que $B = y^{-1} (x ^{-1} B x) y$ (Teorema \ref{teorema:los subgrupos de borel son conjugados y G/B es una variedad proyectiva}). Con lo cual, $xy \in N_G (B) = B$ fuerza que $x \in Q$, y luego $P = Q$.
\end{proof}

En otras palabras, el corolario anterior dice que la cantidad de clases de conjugación de subgrupos parabólicos de $G$ es precisamente la cantidad de subgrupos parabólicos conteniendo un subgrupo de Borel $B$ de $G$ dado.

\begin{corollary}
Sea $B$ un subgrupo de Borel de $G$, y $U = B_u$. Entonces $B = N_G (U)$.
\end{corollary}
\begin{proof}
Sea $N := N_G (U)$. Como $U$ es un subgrupo conexo unipotente maximal de $N^\circ$, contiene un conjugado de cada elemento unipotente de $N^\circ$ (ver \cite[Teorema 22.2]{humphreys2012linearAlgebraicGroups}). Pero $U$ es normal en $N^\circ$, o sea que $N^\circ / U$ consiste de elementos semisimples y por ende es un toro (cf. \cite[Ejercicio 21.2]{humphreys2012linearAlgebraicGroups}). En particular, $N^\circ$ es soluble. Como $B \subset N^\circ$, tenemos que $B = N^\circ$. Sin embargo, $B = N_G (B)$ por el Teorema \ref{teorema:teorema del normalizador}, así que $N^\circ = N$.
\end{proof}




\section{Cosas para agregar}

Suponer que $k$ es un cuerpo algebraicamente cerrado (no necesariamente de caracteristica $0$).

Notacion: $Z ( G), R(G), C_G (x), C_G (S), [G,G], N_G (x), N_G (S), G_x , G\cdot x, B_u, B_s$

definiciones basicas de grupos: solubilidad, subgrupo normal,  nilpotencia, toro maximal, unipotente conexo maximal que es una bandera, que es una bandera completa, una bandera standard (creo que es la bandera de $V = k^n$). Que es un grupo de isotropia. Que es una orbita. Que es el estabilizador. Que es una clase de Conjugación.

definiciones algebraicas: que es un grupo algebraico, que es un grupo algebraico conexo. que es la componente conexa $G^\circ$. Como es la estructura algebraica de un espacio homogeneo. Que es una accion de un grupo algebraico $G$ en una variedad algebraica X. que es la variedad bandera de un espacio vectorial finito $V$, dar como ejemplo la accion natural de $\operatorname{GL}(V)$ allí. Que es un morfismo de grupos algebraicos.

Teoremas de Humphreys: Teorema 11.2, 11.3, (8.3), Teorema 19.3, 19.2, Proposición 17.4, Corolario 16.3. Proposición 7.4B. No copiar solo citar el Teorema 22.4, el 22.3 y el 22.2.

Hacer una observación para no volver a repetirlo que todo grupo algebraico de dimensión $0$ tiene que ser el grupo trivial $\{1\}$.

Ejercicios: 21.5, 21.2









\bibliographystyle{amsalpha}
\bibliography{main}



\end{document}
